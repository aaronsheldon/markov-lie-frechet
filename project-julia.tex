% Julia language highlighting
\lstdefinelanguage{Julia}
{
	keywordsprefix=\@,
	morekeywords={abstract,break,case,catch,const,continue,do,else,elseif,end,
			export,false,for,function,immutable,import,importall,if,in,macro,module,
			otherwise,quote,return,switch,true,try,type,typealias,using,while,begin,
			exit,whos,edit,load,is,isa,isequal,typeof,tuple,ntuple,uid,hash,finalizer,
			convert,promote,subtype,typemin,typemax,realmin,realmax,sizeof,eps,
			promote_type,method_exists,applicable,invoke,dlopen,dlsym,system,error,
			throw,assert,new,Inf,Nan,pi,im,let,bitstype,ccall,baremodule,local,global,
			Bool,Int,Int8,Int16,Int32,Int64,Uint,Uint8,Uint16,Uint32,Uint64,Float32,
			Float64,Complex64,Complex128,Any,Nothing,None,Void},
	sensitive=true,
	alsoother={\$},
	morecomment=[l]\#,
	morecomment=[n]{\#=}{=\#},
	morestring=[s]{"}{"},
	morestring=[m]{'}{'},
}[keywords,comments,strings]

\lstset{
	language         = Julia,
	basicstyle       = \ttfamily,
	numbers          = left, 
	numberstyle      = \small\ttfamily\color{Gray},
	keywordstyle     = \bfseries\color{blue},
	stringstyle      = \color{Maroon},
	commentstyle     = \color{ForestGreen},
	showstringspaces = false,     
	breaklines       = true,    
	tabsize          = 2,
	lineskip         = -1.5pt,
	extendedchars    = true,
	captionpos       = b
}