\chapter{Pad\'{e} Approximation of the Fr\'{e}chet Derivatives of the Exponential Map}
\section{The Gradient}
Moler and Van Loan seminally reviewed algorithms for calculating the matrix exponential in
1978, and revisited that review in 2003 \cite{moler_nineteen_1978,moler_nineteen_2003}. 
Building on the discussions of Moler and Van Loan, Higham established the standard 
implementation of the matrix exponential based on scaling and scaring and Pad\'{e} 
approximation \cite{higham_scaling_2005,higham_functions_2008}. The Higham implementation was
further optimized for $64$ bit architectures by Al-Mohy \cite{al-mohy_new_2009}. In the same
work Al-Mohy developed an algorithm to approximate the derivative of the matrix exponential, 
formulated by taking the derivative of the Pad\'{e} approximation of the matrix exponential, 
and then working out a recursive calculation for the derivatives of matrix 
powers \cite{al-mohy_computing_2009}.

While the derivative of the Pad\'{e} approximation of an analytic function will converge to 
the derivative of the analytic function, it is not true that the derivative of the Pad\'{e} 
approximation of an analytic function is the Pad\'{e} approximation of the derivative of an 
analytic function. In the sense that Pad\'{e} approximations of analytic functions are an 
optimal series of algebraic approximations the 2009 method proposed by Al-Mohy is not 
optimal.

In this chapter we will develop an approximation for the first, and second order Fr\'{e}chet
derivatives of the matrix exponential, by decomposing the derivatives into components that
hold for the commutative condition, and components containing the perturbation due to 
non-commutativity. We will then derive the Pad\'{e} approximation for the non-commutative 
perturbation. We begin by listing the eight forms of the Fr\'{e}chet derivative of 
exponential map, in the direction $\frac{\partial X}{\partial x}$ at the point $X$ in the 
Lie algebra.
\footnote{We have abused and confounded the notations for directional derivatives and 
partial derivatives here by assuming that $X$ is parameterized by $x$ so that $\frac{\partial e^X}{\partial x}$
is the derivative in the direction of change of $x$.}
\footnote{With respect to the adjoint operator, we are using the currying partial 
application notation of $\left[L f\left(\cdotp\right)\right]\left(y\right)$ to indicate the 
application of the operator $L$ to $f\left(x\right)$ followed by evaluation of the result at
$y$.}
{\setlength{\IEEEnormaljot}{18pt}
\begin{IEEEeqnarray*}{rClls}
	\frac{\partial e^X}{\partial x}
		& = & e^X \left[\int_0^1 e^{- s\operatorname{ad}_{X} \cdotp} ds \right] \left(\frac{\partial X}{\partial x}\right) & & \\
		& = & e^X \left[\frac{1 - e^{-\operatorname{ad}_X \cdotp}}{\operatorname{ad}_X \cdotp}\right]\left(\frac{\partial X}{\partial x}\right) & \smash{\left. \IEEEstrut[4\jot] \right\rbrace} & \text{left recursive} \\
		& = & e^X \left[\sum_{n=0}^{\infty} \frac{\left(-1\right)^n}{\left(n+1\right)!} \operatorname{ad}_X^n \cdotp \right] \left(\frac{\partial X}{\partial x}\right) & & \\
		& = & \left[ \frac{\operatorname{ad}_{e^X} \cdotp}{\operatorname{ad}_X \cdotp} \right]\left(\frac{\partial X}{\partial x}\right) & & \text{adjoint ratio} \\
		& = & e^{\frac{1}{2}X} \left[ \frac{e^{\frac{1}{2}\operatorname{ad}_X \cdotp } - e^{-\frac{1}{2}\operatorname{ad}_X \cdotp}}{\operatorname{ad}_X \cdotp} \right]\left(\frac{\partial X}{\partial x}\right) e^{\frac{1}{2}X} & & \text{hyperbolic} \\
		& = & \left[ \int_0^1 e^{s \operatorname{ad}_{X} \cdotp} ds \right]\left(\frac{\partial X}{\partial x}\right) e^X & &\\
		& = & \left[ \frac{e^{\operatorname{ad}_X \cdotp} - 1}{\operatorname{ad}_X \cdotp} \right] \left(\frac{\partial X}{\partial x}\right) e^X & \smash{\left. \IEEEstrut[4\jot] \right\rbrace} & \text{right recursive}\\
		& = & \left[\sum_{n=0}^{\infty} \frac{1}{\left(n+1\right)!} \operatorname{ad}_X^n \cdotp \right] \left(\frac{\partial X}{\partial x} \right) e^X & &
\end{IEEEeqnarray*}}
The last equality demonstrates the non-commutative perturbation term most clearly; where the
first multiplicative factor in the derivative accounts for the lack of commutativity between
$X$ and $\frac{\partial X}{\partial x}$, and the last term resembles the derivative in the
commutative case. This can be seen clearly when considering the condition $\left[X,\frac{\partial X}{\partial x}\right]=0$ 
in which case $\frac{\partial e^X}{\partial x} = \frac{\partial X}{\partial x} e^X$. 

Even though the multiplicative factorization provides a transparent representation of the 
computational terms it is still far from optimal; because, when compared to matrix addition, 
matrix multiplication is both computationally more expensive, and less numerically stable. 
The numerical stability, and efficiency can be improved by decomposing the first 
multiplicative factor into a linear sum of the non-commutative perturbation term, which will 
reduce to $0$ when $\left[X,\frac{\partial X}{\partial x}\right]=0$, and an invariant term 
that contains the commutative relationship for all $X$.
\begin{IEEEeqnarray*}{rCl}
	\frac{\partial e^X}{\partial x}
		& = & \left[\frac{e^{\operatorname{ad}_X \cdotp} - 1 - \operatorname{ad}_X \cdotp }{\operatorname{ad}_X^2 \cdotp} \right] \left(\operatorname{ad}_X \frac{\partial X}{\partial x} \right) e^X + \frac{\partial X}{\partial x} e^X\\
		& = & \underbrace{\left[\sum_{n=0}^{\infty} \frac{1}{\left(n+2\right)!} \operatorname{ad}_X^n \cdotp \right] \left(\operatorname{ad}_X \frac{\partial X}{\partial x} \right)}_{\text{non-commutative anomaly}} e^X + \underbrace{\frac{\partial X}{\partial x}}_{\text{invariant}} e^X
\end{IEEEeqnarray*}
Formally the infinite series in the non-commutative perturbation is related to the lower 
incomplete gamma function $\gamma\left(n,x\right)$. This can be seen by considering the 
general case when the offset of $2$ in the factorial is allowed to be any natural number $n$, 
and then restating the sum in terms of a truncated exponential series.
\begin{IEEEeqnarray*}{rCl}
	\sum_{m=0}^{\infty} \frac{x^m}{\left(m+n\right)!}
		& = & \frac{1}{x^n} \sum_{m=n}^{\infty} \frac{x^m}{m!}\\
		& = & \frac{1}{x^n} \left(e^x - \sum_{m=0}^{n-1}\frac{x^m}{m!} \right)\\
		& = & \frac{1}{x^n} \left(e^x - e^x \frac{\Gamma\left(n,x\right)}{\Gamma\left(n\right)}\right)\\
		& = & \frac{e^x}{\left(n-1\right)! x^n} \left(\int_0^{\infty} t^{n-1}e^{-t}dt - \int_x t^{n-1} e^{-t}dt \right)\\
		& = & \frac{e^x}{\left(n-1\right)! x^n} \int_0^x t^{n-1}e^{-t}dt\\
		& = & \frac{e^x}{\left(n-1\right)! x^n} \gamma\left(n,x\right)
\end{IEEEeqnarray*}
The non-commutative perturbation series is linear in $\frac{\partial X}{\partial x}$  and a 
Taylor series in the powers of $\operatorname{ad}_X \cdotp$. Thus any computation of an 
approximation will be in the powers of $\operatorname{ad}_X \cdotp$. As was discussed in the 
Moler and Van Loan \cite{moler_nineteen_1978,moler_nineteen_2003}, naive computation of the 
Taylor series itself results in an approximation that will converge slowly, requiring a
larger number of powers to be computed before the threshold of floating point error is 
reached. Pad\'{e} approximation by rational functions remedy this problem, by offering 
convergence to the threshold of floating point error in smaller powers, and fewer 
computational steps.

However the question remains, given that $\frac{e^{x} -1 - x}{x^2}$ is a rational 
perturbation of $e^x$, why not simply reuse the polynomials of the Pad\'{e} approximation of 
the exponential function to compute new polynomials for a rational approximation of the 
non-commutative perturbation Taylor series. This method has two shortcomings: first, the 
approximation found in this manner is not itself a Pad\'{e} approximation of the anomaly 
Taylor series, and so is not bound by the same theoretical asymptotic results as Pad\'{e} 
approximations; second, computation by $\frac{e^{x} - 1 - x}{x^2}$ suffers from the same 
floating point errors near $0$ as naive computation of $e^x - 1$ by first computing $e^x$ 
and then subtracting $1$.

While it is clear that $\frac{e^x}{x^2}\gamma\left(2,x\right) : x \mapsto \frac{e^{x} -1 - x}{x^2}$ 
is analytic for $x \in \mathbb{R}$ or $x \in \mathbb{C}$, and thus can be approximated by a 
Pad\'{e} series with coefficients in $\mathbb{C}$; that the rational approximation with the 
same coefficients can be extended to $\operatorname{ad}_X \cdotp$ requires more careful 
consideration.

When $X$ is in $\mathfrak{st}(\hat{\mathbbm{1}})$ the adjoint operator $\operatorname{ad}_X \cdotp$
is a linear endomorphism on the vector space $\mathfrak{st}(\hat{\mathbbm{1}})$; and so 
belongs to the algebra of general linear operators $GL\left(\mathfrak{st}(\hat{\mathbbm{1}})\right)$. 
For a Pad\'{e} approximation with numerator polynomial $P\left(X\right)$ and denominator 
polynomial $Q\left(X\right)$, that $\operatorname{ad}_X \cdotp \in GL\left(\mathfrak{st}(\hat{\mathbbm{1}})\right)$
implies that $P\left(\operatorname{ad}_X \cdotp\right), Q\left(\operatorname{ad}_X \cdotp\right) \in GL\left(\mathfrak{st}(\hat{\mathbbm{1}})\right)$.
It follows that when a solution $Y$ to $\left[P\left(\operatorname{ad}_X \cdotp\right)\right] \left(\frac{\partial X}{\partial x}\right) = \left[Q\left(\operatorname{ad}_X \cdotp\right)\right] \left(Y\right) $
exists it is guaranteed to belong to $\mathfrak{st}(\hat{\mathbbm{1}})$, because $\frac{\partial X}{\partial x} \in \mathfrak{st}(\hat{\mathbbm{1}})$.

The remaining question is whether or not $Y$, a solution to $\left[P\left(\operatorname{ad}_X \cdotp\right)\right] \left(\frac{\partial X}{\partial x}\right) = \left[Q\left(\operatorname{ad}_X \cdotp\right)\right] \left(Y\right) $,
is an approximation of $\left[\sum_{n=0}^{\infty} \frac{1}{\left(n+2\right)!} \operatorname{ad}_X^n \cdotp \right] \left(\operatorname{ad}_X \frac{\partial X}{\partial x} \right)$?
Loosely, $\mathfrak{st}(\hat{\mathbbm{1}})$ is a normed vector space in the usual sense, so 
the convergence of the Pad\'{e} approximations that apply in scalar spaces carry over. Thus given
a sequence of Pad\'{e} approximations $Y_{mn}$ that solve $\left[P_n\left(\operatorname{ad}_X \cdotp\right)\right] \left(\frac{\partial X}{\partial x}\right) = \left[Q_m\left(\operatorname{ad}_X \cdotp\right)\right] \left(Y_{mn}\right) $,
convergence $Y_{mn} \rightarrow \left[\sum_{n=0}^{\infty} \frac{1}{\left(n+2\right)!} \operatorname{ad}_X^n \cdotp \right] \left(\operatorname{ad}_X \frac{\partial X}{\partial x} \right)$
is assured. 

Recapitulating $\left[n/m\right]_f\left(x\right)$ Pad\'{e} approximations, we seek a 
rational polynomial approximation to the Taylor series.
\begin{IEEEeqnarray*}{rCl}
	\frac{P_n\left(x\right)}{Q_m\left(x\right)}
		& =       & \frac{p_0 + p_1 x + \cdots + p_n x^n}{1 + q_1 x + \cdots + q_m x^m}\\
		& \approx & \sum_{n=0}^\infty \frac{1}{\left(n+2\right)!}x^n
\end{IEEEeqnarray*}
Such that the first $k \le n+m$ derivatives of the rational polynomial approximation evaluated at
$x=0$ equal the first $k \le n+m$ coefficients of the Taylor series.
\begin{IEEEeqnarray*}{rCl}
	\left.\frac{d^k}{d x^k}\frac{P_n\left(x\right)}{Q_m\left(x\right)}\right|_{x=0}
		& = & \frac{1}{\left(k+2\right)!}
\end{IEEEeqnarray*}
The symbolic computation of the exact rational coefficients of the $\left[13/14\right]_f\left(x\right)$ 
Pad\'{e} approximation was carried out in Julia programming language \cite{bezanson_julia:_2014}
using big integer mathematics, and the Polynomials package. The results of the computation 
are displayed in table \ref{tab:perturbation}. The order of the Pad\'{e} approximation 
was chosen so that the largest denominator in the rational coefficients of the numerator
polynomial $\left(7244400176133120000\right)$, and the largest denominator in the rational 
coefficients of the denominator polynomial $\left(6761440164390912000\right)$ were the 
largest integers less than the largest $64$ bit signed integer $\left(9223372036854775807\right)$.
This choice of approximation will need further research to better optimize.

To make use of the Pad\'{e} approximation we need to be able to compute powers of $\operatorname{ad}_X \cdotp$.
This can be accomplished through the Kronecker representation of $\operatorname{ad}_X \cdotp$, 
which requires representing the matrices of $\mathfrak{st}(\hat{\mathbbm{1}})$ as vectors. 
The vector representation of a matrix is achieved by the matrix reshaping  operator $\operatorname{vec}\left(Y\right) = \vec{y}$, 
which forms a vector $\vec{y}$ by concatenation of the columns of $Y$, called the 
vectorization of the matrix. We denote the inverse operator to vectorization $\operatorname{mat}\left(\vec{y}\right) = \operatorname{vec}^{-1}\left(\vec{y}\right) = Y$,
which reshapes a vector, of $n^2$ entries, into an $n \times n$ matrix.

After juggling the indexes of $\operatorname{vec}\left(\frac{\partial X}{\partial x}\right)$, 
the Kronecker representation of $\operatorname{ad}_X \frac{\partial X}{\partial x}$ follows
as
\begin{IEEEeqnarray*}{rCl}
	\operatorname{ad}_X \frac{\partial X}{\partial x} 
		& = & \operatorname{mat}\left( \left(I \otimes X - X^\dagger \otimes I \right) \operatorname{vec}\left(\frac{\partial X}{\partial x}\right)\right)
\end{IEEEeqnarray*}
Proceeding by induction we find that 
\begin{IEEEeqnarray*}{rCl}
	\operatorname{ad}_X^n \frac{\partial X}{\partial x} 
		& = & \operatorname{mat}\left(\left(I \otimes X - X^\dagger \otimes I \right)^n\operatorname{vec}\left(\frac{\partial X}{\partial x}\right)\right)
\end{IEEEeqnarray*}
It follows that $\frac{\partial e^X}{\partial x}$ can be computed by
\begin{IEEEeqnarray*}{rCl}
	\frac{\partial e^X}{\partial x}
		& = & \operatorname{mat}\left(\sum_{n=0}^{\infty} \frac{1}{\left(n+2\right)!} \left(I \otimes X - X^\dagger \otimes I \right)^{n+1} \operatorname{vec}\left(\frac{\partial X}{\partial x}\right)\right) e^X + \frac{\partial X}{\partial x} e^X
\end{IEEEeqnarray*}
And thus can be approximated by
\begin{IEEEeqnarray*}{rCl}
	\frac{\partial e^X}{\partial x}
		& \approx & \operatorname{mat}\left(\frac{P_n\left(I \otimes X - X^\dagger \otimes I\right)}{Q_m \left(I \otimes X - X^\dagger \otimes I\right)} \left(I \otimes X - X^\dagger \otimes I \right)\operatorname{vec}\left(\frac{\partial X}{\partial x}\right) \right) e^X + \frac{\partial X}{\partial x} e^X
\end{IEEEeqnarray*}
We can summarize this work in an algorithm to compute the non-commutative perturbation \ref{alg:perturbation}.
This algorithm servers as a sketch only, highlighting only the novel elements developed in
this section. Many additional optimizations could be implemented including minimizing memory 
assignments by carrying out in place computations, conditioning matrices to improve
numerical stability, and using recursive squaring and summing methods to efficiently compute
the matrix polynomials. A point of concern is the first assignment of the Kronecker product, 
which results in a quadratic increase in the amount of memory used. Finally, an algorithm to 
compute the gradient of the matrix exponential can be formulated \ref{alg:gradient}.
\section{The Hessian}
Assuming $X$ parameterized by $x,y \in \mathbb{R}$ is analytic, or at least twice 
differentiable the Hessian of $e^X$ exists and depends on three tangent matrices, $\frac{\partial X}{\partial x}$,
$\frac{\partial X}{\partial y}$, and $\frac{\partial^2 X}{\partial x \partial y}$. In 
general none of these tangent matrices need commute with each other. Even the second 
derivatives $\frac{\partial^2 X}{\partial x^2}$, will not generally commute with the first 
derivate $\frac{\partial X}{\partial x}$. While the additional terms complicate the 
computations, they do not lead to intractable results in the same way that finding a closed
form for $e^X$ in dimensions greater than $4$ becomes intractable. Unfortunately in the most 
general case when $X$ neither commutes with $\frac{\partial X}{\partial x}$, nor $\frac{\partial X}{\partial y}$ 
we will find that we have to compute the Taylor expansion of a bilinear form, which is not 
susceptible to Pad\'{e} approximation.

To proceed we need a pair of results focused on the binomial combinatorics of adjoints. We
will use these results in developing the bilinear non-commutative perturbation of the
Hessian.
\begin{lemma}
	For differentiable matrix function $X$ parameterized by $x \in \mathbb{R}$, any matrices $A,B$,
	and integer $n \ge 0$
	\begin{IEEEeqnarray*}{rCl}
		\left[\frac{\partial}{\partial x}\operatorname{ad}_X^n \cdot\right]\left( A\right)
			& = & \sum_{k=1}^n \left(\operatorname{ad}_X^{k-1} A \right)\left(\operatorname{ad}_{\frac{\partial X}{\partial x}} A\right)  \left(\operatorname{ad}_X^{n-k} A \right)\\
		\operatorname{ad}_X^n AB
			& = & \sum_{k=0}^n \binom{n}{k} \left(\operatorname{ad}_X^k A \right)\left(\operatorname{ad}_X^{n-k} B \right)\\
		\operatorname{ad}_X^n \left[A,B\right]
			& = & \sum_{k=0}^n \binom{n}{k} \left[\operatorname{ad}_X^k A , \operatorname{ad}_X^{n-k} B \right]
	\end{IEEEeqnarray*}
\end{lemma}
\begin{IEEEproof}
	For each of the equalities we have:
	\begin{enumerate}
		\item Proceed by induction on $n$.
		\item Proceed by induction on $n$.
		\item Take the antisymmetric difference of previous equality.\hfill\IEEEQEDhere
	\end{enumerate}
\end{IEEEproof}
We will also find use of the following corollary to the binomial theorem.
\begin{corollary}
	For any $n,m \ge 0$
	\begin{IEEEeqnarray*}{rCl}
		\sum_{k=m}^{n+m} \binom{k}{m}
			& = & \binom{n+m+1}{n}
	\end{IEEEeqnarray*}
\end{corollary}
\begin{IEEEproof}
	Proceed by induction on $n$.\hfill\IEEEQEDhere
\end{IEEEproof}
In the next step of developing the Hessian we derive the Taylor series for the bilinear 
non-commutative perturbation. Unfortunately because this form is a bilinear map it does
not admit the formulation of Pad\'{e} approximation.
\begin{corollary}
	For any differentiable matrix function $X$ parameterized by $x,y \in \mathbb{R}$
	\begin{IEEEeqnarray*}{rCl}
		\IEEEeqnarraymulticol{3}{l}
		{
			\left[\frac{\partial}{\partial x} \sum_{n=1}^\infty \frac{1}{\left(n+1\right)!} \operatorname{ad}_X^n \cdotp \right]\left(\frac{\partial X}{\partial y}\right)
			+ \left[\frac{\partial}{\partial y} \sum_{n=1}^\infty \frac{1}{\left(n+1\right)!} \operatorname{ad}_X^n \cdotp \right]\left(\frac{\partial Y}{\partial x}\right)
		}\\\quad
			& = & \sum_{n \ge m \ge 0} \frac{1}{\left(n+2\right)!} \left(\binom{n+1}{m+1} - \binom{n+1}{m} \right) \left[\operatorname{ad}_X^{m} \frac{\partial X}{\partial x} ,\operatorname{ad}_X^{n-m} \frac{\partial X}{\partial y} \right]
	\end{IEEEeqnarray*}
\end{corollary}
\begin{IEEEproof}
	Apply the previous results in the order they were stated to the symmetric sum of the two
	terms.
	\begin{IEEEeqnarray*}{rCl}
		\IEEEeqnarraymulticol{3}{l}
		{
			\left[\frac{\partial}{\partial x} \sum_{n=1}^\infty \frac{1}{\left(n+1\right)!} \operatorname{ad}_X^n \cdotp \right]\left(\frac{\partial X}{\partial y}\right)
			+ \left[\frac{\partial}{\partial y} \sum_{n=1}^\infty \frac{1}{\left(n+1\right)!} \operatorname{ad}_X^n \cdotp \right]\left(\frac{\partial Y}{\partial x}\right)
		}\\\quad
			& = & \sum_{n=1}^\infty \frac{1}{\left(n+1\right)!} \sum_{k=1}^n \operatorname{ad}_X^{k-1} \operatorname{ad}_{\frac{\partial X}{\partial x}} \operatorname{ad}_X^{n-k} \frac{\partial X}{\partial y}\\
			&   & +\: \sum_{n=1}^\infty \frac{1}{\left(n+1\right)!} \sum_{k=1}^n \operatorname{ad}_X^{k-1} \operatorname{ad}_{\frac{\partial X}{\partial y}} \operatorname{ad}_X^{n-k} \frac{\partial X}{\partial x}\\
			& = & \sum_{n=1}^\infty \frac{1}{\left(n+1\right)!} \sum_{k=1}^n \operatorname{ad}_X^{k-1} \left[ \frac{\partial X}{\partial x} ,\operatorname{ad}_X^{n-k} \frac{\partial X}{\partial y} \right]\\
			&   & +\: \sum_{n=1}^\infty \frac{1}{\left(n+1\right)!} \sum_{k=1}^n \operatorname{ad}_X^{k-1} \left[ \frac{\partial X}{\partial y} ,\operatorname{ad}_X^{n-k} \frac{\partial X}{\partial x} \right]\\
			& = & \sum_{n=1}^\infty \frac{1}{\left(n+1\right)!} \sum_{k=1}^n \sum_{m=0}^{k-1} \binom{k-1}{m} \left[\operatorname{ad}_X^{m} \frac{\partial X}{\partial x} ,\operatorname{ad}_X^{n-m-1} \frac{\partial X}{\partial y} \right]\\
			&   & +\: \sum_{n=1}^\infty \frac{1}{\left(n+1\right)!} \sum_{k=1}^n \sum_{m=0}^{k-1} \binom{k-1}{m} \left[\operatorname{ad}_X^{m} \frac{\partial X}{\partial y} ,\operatorname{ad}_X^{n-m-1} \frac{\partial X}{\partial x} \right]\\
			& = & \sum_{n=0}^\infty \frac{1}{\left(n+2\right)!} \sum_{k=0}^n \sum_{m=0}^{k} \binom{k}{m} \left[\operatorname{ad}_X^{m} \frac{\partial X}{\partial x} ,\operatorname{ad}_X^{n-m} \frac{\partial X}{\partial y} \right]\\
			&   & +\: \sum_{n=0}^\infty \frac{1}{\left(n+2\right)!} \sum_{k=0}^n \sum_{m=0}^{k} \binom{k}{m} \left[\operatorname{ad}_X^{m} \frac{\partial X}{\partial y} ,\operatorname{ad}_X^{n-m} \frac{\partial X}{\partial x} \right]\\
			& = & \sum_{n,m \ge 0}^\infty \frac{1}{\left(n+m+2\right)!} \left[\operatorname{ad}_X^{m} \frac{\partial X}{\partial x} ,\operatorname{ad}_X^{n} \frac{\partial X}{\partial y} \right] \sum_{k=m}^{n+m} \binom{k}{m}\\
			&   & +\: \sum_{n,m \ge 0}^\infty \frac{1}{\left(n+m+2\right)!} \left[\operatorname{ad}_X^{n} \frac{\partial X}{\partial y} ,\operatorname{ad}_X^{m} \frac{\partial X}{\partial x} \right] \sum_{k=n}^{n+m} \binom{k}{n}\\
			& = & \sum_{n,m \ge 0}^\infty \frac{1}{\left(n+m+2\right)!} \left(\binom{n+m+1}{n} - \binom{n+m+1}{m} \right) \left[\operatorname{ad}_X^{m} \frac{\partial X}{\partial x} ,\operatorname{ad}_X^{n} \frac{\partial X}{\partial y} \right]\\
			& = & \sum_{n \ge m \ge 0} \frac{1}{\left(n+2\right)!} \left(\binom{n+1}{m+1} - \binom{n+1}{m} \right) \left[\operatorname{ad}_X^{m} \frac{\partial X}{\partial x} ,\operatorname{ad}_X^{n-m} \frac{\partial X}{\partial y} \right]\\
			& = & \sum_{n=0}^\infty \frac{1}{\left(n+2\right)!} F_n
	\end{IEEEeqnarray*}
	Where we have defined $F_n$ recursively as:
	\begin{IEEEeqnarray*}{rCl}
		F_0 & = & 0\\
		F_{n+1} & = &  \left[\frac{\partial X}{\partial x},\operatorname{ad}_X^{n+1} \frac{\partial X}{\partial y}\right] + \operatorname{ad}_X F_n - \left[\operatorname{ad}_X^{n+1} \frac{\partial X}{\partial x},\frac{\partial X}{\partial y}\right]
	\end{IEEEeqnarray*}
	This recursive calculation is illustrate in figure \ref{fig:pascaldivergence}. The proof 
	of which is found by carrying out induction on $n$.\hfill\IEEEQEDhere
\end{IEEEproof}
There are three special cases that simplify the calculation of Taylor series considerably.
\begin{IEEEeqnarray*}{rCl}
	\sum_{n=0}^\infty \frac{ F_n}{\left(n+2\right)!}
	& = & 
		\begin{cases}
			0
				& \operatorname{ad}_X \frac{\partial X}{\partial x} = 0 \text{ and } \operatorname{ad}_X \frac{\partial X}{\partial y} = 0\\[1ex]
			\left[\frac{\partial X}{\partial x}, \sum_{n=0}^\infty \frac{n}{\left(n+2\right)!} \operatorname{ad}_X^n \frac{\partial X}{\partial y}\right]
				& \operatorname{ad}_X  \frac{\partial X}{\partial x} = 0\\[1ex]
			\left[\frac{\partial X}{\partial y}, \sum_{n=0}^\infty \frac{n}{\left(n+2\right)!} \operatorname{ad}_X^n \frac{\partial X}{\partial x}\right]
				& \operatorname{ad}_X  \frac{\partial X}{\partial y} = 0
		\end{cases}
\end{IEEEeqnarray*}
The Taylor series in the last two cases does admit a Pad\'{e} approximation, by the same
reasoning as presented in the previous section. In particular we can further reduce the
Taylor series.
\begin{IEEEeqnarray*}{rCl}
	\sum_{n=0}^\infty \frac{n}{\left(n+2\right)!} x^n
		& = & \sum_{n=1}^\infty \frac{n}{\left(n+2\right)!} x^n\\
		& = & \left(\sum_{n=0}^\infty \frac{n+1}{\left(n+3\right)!} x^n\right)x
\end{IEEEeqnarray*}
Using the same criteria as in the previous section yields a $\left[12/14\right]_f\left(x\right)$
Pad\'{e} approximation for $f\left(x\right) = \sum_{n=0}^\infty \frac{n+1}{\left(n+3\right)!} x^n$.
The coefficients of the Pad\'{e} approximation are summarized in table \ref{tab:bilinear}. 
This approximation is then used in a pair of branches in the calculation of the bilinear 
non-commutative perturbation to the Hessian of the matrix exponential \ref{alg:bilinear}.

Assuming that $\frac{\partial^2 X}{\partial x \partial y},\frac{\partial^2 X}{\partial y \partial x}$ 
are continuous we have, by corollary to Clairaut's theorem, that $\frac{\partial^2 e^X}{\partial x \partial y} = \frac{\partial^2 e^X}{\partial y \partial x}$.
We can then compute the Hessian by symmetrizing the partial differential so that 
antisymmetric terms cancel out.
\begin{IEEEeqnarray*}{rCl}
	\frac{1}{2} \left(\frac{\partial^2 e^X}{\partial x \partial y} + \frac{\partial^2 e^X}{\partial y \partial x}\right)
		& = & \frac{1}{2} \frac{\partial}{\partial x} \left[\sum_{n=0}^{\infty} \frac{1}{\left(n+1\right)!} \operatorname{ad}_X^n \cdotp \right] \left(\frac{\partial X}{\partial y} \right) e^X\\
		&   & +\: \frac{1}{2} \frac{\partial}{\partial y} \left[\sum_{n=0}^{\infty} \frac{1}{\left(n+1\right)!} \operatorname{ad}_X^n \cdotp \right] \left(\frac{\partial X}{\partial x} \right) e^X\\
		& = & \frac{1}{2} \left[\sum_{n=0}^{\infty} \frac{1}{\left(n+1\right)!} \operatorname{ad}_X^n \cdotp \right] \left(\frac{\partial X}{\partial y} \right) \left[\sum_{n=0}^{\infty} \frac{1}{\left(n+1\right)!} \operatorname{ad}_X^n \cdotp \right] \left(\frac{\partial X}{\partial x} \right) e^X\\
		&   & +\: \frac{1}{2} \left[\sum_{n=0}^{\infty} \frac{1}{\left(n+1\right)!} \operatorname{ad}_X^n \cdotp \right] \left(\frac{\partial^2 X}{\partial x \partial y} \right) e^X\\
		&   & \quad +\: \frac{1}{2} \left[ \frac{\partial}{\partial x} \sum_{n=0}^{\infty} \frac{1}{\left(n+1\right)!} \operatorname{ad}_X^n \cdotp \right] \left(\frac{\partial X}{\partial y}\right) e^X\\
		&   & +\: \frac{1}{2} \left[\sum_{n=0}^{\infty} \frac{1}{\left(n+1\right)!} \operatorname{ad}_X^n \cdotp \right] \left(\frac{\partial X}{\partial x} \right) \left[\sum_{n=0}^{\infty} \frac{1}{\left(n+1\right)!} \operatorname{ad}_X^n \cdotp \right] \left(\frac{\partial X}{\partial y} \right) e^X\\
		&   & \quad +\: \frac{1}{2} \left[\sum_{n=0}^{\infty} \frac{1}{\left(n+1\right)!} \operatorname{ad}_X^n \cdotp \right] \left(\frac{\partial^2 X}{\partial y \partial x} \right) e^X\\
		&   & \quad\quad +\: \frac{1}{2} \left[\frac{\partial}{\partial y} \sum_{n=0}^{\infty} \frac{1}{\left(n+1\right)!} \operatorname{ad}_X^n \cdotp \right] \left(\frac{\partial X}{\partial x}\right) e^X\\
		& = & \left[\sum_{n=0}^{\infty} \frac{1}{\left(n+1\right)!} \operatorname{ad}_X^n \cdotp \right] \left(\frac{\partial^2 X}{\partial x \partial y} \right) e^X\\
		&   & +\: \frac{1}{2} \left[\sum_{n=0}^{\infty} \frac{1}{\left(n+1\right)!} \operatorname{ad}_X^n \cdotp \right] \left(\frac{\partial X}{\partial y} \right) \left[\sum_{n=0}^{\infty} \frac{1}{\left(n+1\right)!} \operatorname{ad}_X^n \cdotp \right] \left(\frac{\partial X}{\partial x} \right) e^X\\
		&   & +\: \frac{1}{2} \left[\sum_{n=0}^{\infty} \frac{1}{\left(n+1\right)!} \operatorname{ad}_X^n \cdotp \right] \left(\frac{\partial X}{\partial x} \right) \left[\sum_{n=0}^{\infty} \frac{1}{\left(n+1\right)!} \operatorname{ad}_X^n \cdotp \right] \left(\frac{\partial X}{\partial y} \right) e^X\\
		&   & +\: \frac{1}{2} \sum_{n \ge m \ge 0} \frac{1}{\left(n+2\right)!} \left(\binom{n+1}{m+1} - \binom{n+1}{m} \right) \left[\operatorname{ad}_X^{m} \frac{\partial X}{\partial x} ,\operatorname{ad}_X^{n-m} \frac{\partial X}{\partial y} \right]
\end{IEEEeqnarray*}
To flush out the final algorithm for the Hessian lets consider each summand in the last 
equality individually. The first term involving $\frac{\partial^2 X}{\partial x \partial y}$
can be calculated using the non-commutative perturbation algorithm developed in the 
preceding section
\begin{IEEEeqnarray*}{rCl}
	\left[\sum_{n=0}^{\infty} \frac{1}{\left(n+1\right)!} \operatorname{ad}_X^n \cdotp \right] \left(\frac{\partial^2 X}{\partial x \partial y} \right)
		& = & \left[\sum_{n=0}^{\infty} \frac{1}{\left(n+2\right)!} \operatorname{ad}_X^n \cdotp \right] \left(\operatorname{ad}_X \frac{\partial^2 X}{\partial x \partial y} \right) + \frac{\partial^2 X}{\partial x \partial y}
\end{IEEEeqnarray*}
The next two summands together are the Poisson bracket of the non-commutative perturbations
the gradients $\frac{\partial e^X}{\partial x}$, and $\frac{\partial e^X}{\partial y}$. The
final summand is the bilinear non-commutative perturbation. Taken together the algorithm for
the Hessian is then a sequence of calls to the non-commutative perturbation and the bilinear
non-commutative perturbation; as outlined in algorithm \ref{alg:hessian}.

As in the previous section the algorithm we have developed for the Hessian of the matrix
exponential is merely a starting point for further optimizations. In particular the bilinear
non-commutative perturbation needs attention to see if the Taylor series i susceptible to
further efficiency gains. As well the Pad\'{e} approximation needs further refinement.
Nevertheless both of the algorithms for the non-commutative perturbation and the bilinear
non-commutative perturbation are stable with respect to the the stochastic contraction
Lie algebra $\mathfrak{st}^{+}(\hat{\mathbbm{1}})$; in the sense that if $X, \frac{\partial X}{\partial x}, \frac{\partial X}{\partial y}, \frac{\partial^2 X}{\partial x \partial y} \in \mathfrak{st}^{+}(\hat{\mathbbm{1}})$,
then the result of the algorithms will be in $\mathfrak{st}^{+}(\hat{\mathbbm{1}})$.
\clearpage
\section{Figures and Illustrations}
\begin{longtable}{r r r}
	\caption{Pad\'{e} Approximation of $\sum_{n=0}^\infty \frac{1}{\left(n+2\right)!} x^n$}
	\label{tab:perturbation}\\
	\multicolumn{1}{l}{Degree} & \multicolumn{1}{l}{Numerator} & \multicolumn{1}{l}{Denominator}\\
	\hline
	\endfirsthead
	\caption*{Continued from previous page.}\\
	\multicolumn{1}{l}{Degree} & \multicolumn{1}{l}{Numerator} & \multicolumn{1}{l}{Denominator}\\
	\hline
	\endhead
	\caption*{Continued on next page.}
	\endfoot
	\caption*{The exact rational coefficients of the numerator and denominator polynomials of the $\left[ 13/14 \right]_f\left(x\right)$ Pad\'{e} approximation of $f\left(x\right)=\sum_{n=0}^\infty \frac{1}{\left(n+2\right)!} x^n$; symbolically computed.}
	\endlastfoot
	$x^{0}$ & $\frac{1}{2}$ & $\frac{1}{1}$\\
	$x^{1}$ & $\frac{-13}{174}$ & $\frac{-14}{29}$\\
	$x^{2}$ & $\frac{1}{58}$ & $\frac{13}{116}$\\
	$x^{3}$ & $\frac{-11}{7830}$ & $\frac{-13}{783}$\\
	$x^{4}$ & $\frac{11}{75168}$ & $\frac{11}{6264}$\\
	$x^{5}$ & $\frac{-11}{1461600}$ & $\frac{-11}{78300}$\\
	$x^{6}$ & $\frac{1}{2192400}$ & $\frac{11}{1252800}$\\
	$x^{7}$ & $\frac{-1}{64832400}$ & $\frac{-11}{25212600}$\\
	$x^{8}$ & $\frac{1}{1728864000}$ & $\frac{1}{57628800}$\\
	$x^{9}$ & $\frac{-1}{79873516800}$ & $\frac{-1}{1815307200}$\\
	$x^{10}$ & $\frac{1}{3594308256000}$ & $\frac{1}{72612288000}$\\
	$x^{11}$ & $\frac{-1}{295931379744000}$ & $\frac{-1}{3793992048000}$\\
	$x^{12}$ & $\frac{1}{28409412455424000}$ & $\frac{1}{273167427456000}$\\
	$x^{13}$ & $\frac{-1}{7244400176133120000}$ & $\frac{-1}{30185000733888000}$\\
	$x^{14}$ & & $\frac{1}{6761440164390912000}$
\end{longtable}
\begin{algorithm}[!ht]
	\caption[Non-commutative Perturbation of $\frac{\partial e^X}{\partial x}$]{Numerical calculation the non-commutative perturbation of the gradient of the matrix exponential $\frac{\partial e^X}{\partial x}$.}
	\label{alg:perturbation}
	\begin{algorithmic}[1]
		\Function{PER}{$X$,$D_x$}
			\If{$\left[X,D_x\right] = 0$}
				\State $Y \gets 0$\Comment{Short circuit commutative case}
			\Else
				\State $A_X \gets I \otimes X - X^\dagger \otimes I$\Comment{if $X$ is $n \times n$ the result is $n^2 \times n^2$}
				\State $A_x \gets [X, D_x]$\Comment{Allocates memory}
				\State $\vec{a_x} \gets \operatorname{VEC}\left(A_x\right)$\Comment{Change of indexing}
				\State $P \gets P_{13}\left(A_X\right)$\Comment{Recursive squaring and summing}
				\State $Q \gets Q_{14}\left(A_X\right)$\Comment{Recursive squaring and summing}
				\State Solve for $\vec{y}$: $P \vec{a_x} = Q \vec{y}$\Comment{Call to linear solver}
				\State $Y \gets \operatorname{MAT}\left(\vec{y}\right)$\Comment{Change of indexing}
			\EndIf
			\State \Return $Y$
		\EndFunction
	\end{algorithmic}
\end{algorithm}
\begin{algorithm}[!ht]
	\caption[Gradient of $e^X$]{Numerical calculation of the gradient of the matrix exponential $\frac{\partial e^X}{\partial x}$.}
	\label{alg:gradient}
	\begin{algorithmic}[1]
		\Function{GEX}{$X$,$D_x$}
			\State $E \gets e^X$\Comment{Call to matrix exponential}
			\State $P \gets \operatorname{PER}\left(X,D_x\right)$\Comment{Call to non-commutative perturbation}
			\State \Return $\left(P + D_x\right) E$
		\EndFunction
	\end{algorithmic}
\end{algorithm}
\begin{figure}[!ht]
	\centering
	\begin{tikzpicture}
		\node[text=gray] at (   0,   0) {$  0$};

		\node[text=gray] at (-1/2,  -1) {$  1$};
		\node[text=gray] at ( 1/2,  -1) {$ -1$};

		\node[text=gray] at (  -1,  -2) {$  1$};
		\node[font=\bf ] at (   0,  -2) {$  0$};
		\node[text=gray] at (   1,  -2) {$ -1$};

		\node[text=gray] at (-3/2,  -3) {$  1$};
		\node[font=\bf ] at (-1/2,  -3) {$  1$};
		\node[font=\bf ] at ( 1/2,  -3) {$ -1$};
		\node[text=gray] at ( 3/2,  -3) {$ -1$};

		\node[text=gray] at (  -2,  -4) {$  1$};
		\node[font=\bf ] at (  -1,  -4) {$  2$};
		\node[font=\bf ] at (   0,  -4) {$  0$};
		\node[font=\bf ] at (   1,  -4) {$ -2$};
		\node[text=gray] at (   2,  -4) {$ -1$};

		\node[text=gray] at (-5/2,  -5) {$  1$};
		\node[font=\bf ] at (-3/2,  -5) {$  3$};
		\node[font=\bf ] at (-1/2,  -5) {$  2$};
		\node[font=\bf ] at ( 1/2,  -5) {$ -2$};
		\node[font=\bf ] at ( 3/2,  -5) {$ -3$};
		\node[text=gray] at ( 5/2,  -5) {$ -1$};

		\node[text=gray] at (  -3,  -6) {$  1$};
		\node[font=\bf ] at (  -2,  -6) {$  4$};
		\node[font=\bf ] at (  -1,  -6) {$  5$};
		\node[font=\bf ] at (   0,  -6) {$  0$};
		\node[font=\bf ] at (   1,  -6) {$ -5$};
		\node[font=\bf ] at (   2,  -6) {$ -4$};
		\node[text=gray] at (   3,  -6) {$ -1$};

		\node[text=gray] at (-7/2,  -7) {$  1$};
		\node[font=\bf ] at (-5/2,  -7) {$  5$};
		\node[font=\bf ] at (-3/2,  -7) {$  9$};
		\node[font=\bf ] at (-1/2,  -7) {$  5$};
		\node[font=\bf ] at ( 1/2,  -7) {$ -5$};
		\node[font=\bf ] at ( 3/2,  -7) {$ -9$};
		\node[font=\bf ] at ( 5/2,  -7) {$ -5$};
		\node[text=gray] at ( 7/2,  -7) {$ -1$};

		\node[text=gray] at (  -4,  -8) {$  1$};
		\node[font=\bf ] at (  -3,  -8) {$  6$};
		\node[font=\bf ] at (  -2,  -8) {$ 14$};
		\node[font=\bf ] at (  -1,  -8) {$ 14$};
		\node[font=\bf ] at (   0,  -8) {$  0$};
		\node[font=\bf ] at (   1,  -8) {$-14$};
		\node[font=\bf ] at (   2,  -8) {$-14$};
		\node[font=\bf ] at (   3,  -8) {$ -6$};
		\node[text=gray] at (   4,  -8) {$ -1$};

		\node[text=gray] at (-9/2,  -9) {$  1$};
		\node[font=\bf ] at (-7/2,  -9) {$  7$};
		\node[font=\bf ] at (-5/2,  -9) {$ 20$};
		\node[font=\bf ] at (-3/2,  -9) {$ 28$};
		\node[font=\bf ] at (-1/2,  -9) {$ 14$};
		\node[font=\bf ] at ( 1/2,  -9) {$-14$};
		\node[font=\bf ] at ( 3/2,  -9) {$-28$};
		\node[font=\bf ] at ( 5/2,  -9) {$-20$};
		\node[font=\bf ] at ( 7/2,  -9) {$ -7$};
		\node[text=gray] at ( 9/2,  -9) {$ -1$};

		\node[text=gray] at (  -5, -10) {$\left[\frac{\partial X}{\partial x},\operatorname{ad}_X^{n+1} \frac{\partial X}{\partial y}\right]$};
		\node[text=gray] at (   5, -10) {$-\left[\operatorname{ad}_X^{n+1} \frac{\partial X}{\partial x},\frac{\partial X}{\partial y}\right]$};
	\end{tikzpicture}
	\caption[Pascal's triangle divergence]{Illustration of the calculation of the first 8 rows of coefficients of the divergence of Pascal's triangle; emphasizing the coefficients in $F_n$.}
	\label{fig:pascaldivergence}
\end{figure}
\begin{longtable}{r r r}
	\caption{Pad\'{e} Approximation of $\sum_{n=0}^\infty \frac{n+1}{\left(n+3\right)!} x^n$}
	\label{tab:bilinear}\\
	\multicolumn{1}{l}{Degree} & \multicolumn{1}{l}{Numerator} & \multicolumn{1}{l}{Denominator}\\
	\hline
	\endfirsthead
	\caption*{Continued from previous page.}\\
	\multicolumn{1}{l}{Degree} & \multicolumn{1}{l}{Numerator} & \multicolumn{1}{l}{Denominator}\\
	\hline
	\endhead
	\caption*{Continued on next page.}
	\endfoot
	\caption*{The exact rational coefficients of the numerator and denominator polynomials of the $\left[ 12/14 \right]_f\left(x\right)$ Pad\'{e} approximation of $f\left(x\right)=\sum_{n=0}^\infty \frac{n+1}{\left(n+3\right)!} x^n$; symbolically computed.}
	\endlastfoot
	$x^{0}$ & $\frac{1}{6}$ & $\frac{1}{1}$\\
	$x^{1}$ & $\frac{-237571687770}{2235524846677109}$ & $\frac{-2238375706930349}{4471049693354218}$\\
	$x^{2}$ & $\frac{75899205239671}{22355248466771090}$ & $\frac{7}{58}$\\
	$x^{3}$ & $\frac{-6158055716087}{2897240201293533264}$ & $\frac{-172676164807703}{9286026286197222}$\\
	$x^{4}$ & $\frac{14904653897243}{751136348483508624}$ & $\frac{989302956681193}{482873366882255544}$\\
	$x^{5}$ & $\frac{-2677622343899}{225340904545052587200}$ & $\frac{-317642104732513}{1857205257239444400}$\\
	$x^{6}$ & $\frac{15554305889797}{347668824155223991680}$ & $\frac{1079293136317583}{96574673376451108800}$\\
	$x^{7}$ & $\frac{-1086326787569}{44424349753167510048000}$ & $\frac{-18091761597171823}{31097044827217257033600}$\\
	$x^{8}$ & $\frac{23665068673453}{586401416741811132633600}$ & $\frac{215879922742633}{8884869950633502009600}$\\
	$x^{9}$ & $\frac{-579062082583}{31665676504057801162214400}$ & $\frac{-129862571470787}{159927659111403036172800}$\\
	$x^{10}$ & $\frac{1006334079353}{82096198343853558568704000}$ & $\frac{239909667097237}{11194936137798212532096000}$\\
	$x^{11}$ & $\frac{-129504852863}{36499969783677292139645798400}$ & $\frac{-290350506989849}{668497615085664691202304000}$\\
	$x^{12}$ & $\frac{18809879890171}{32849972805309562925681218560000}$ & $\frac{539968479224557}{84230699500793751091490304000}$\\
	$x^{13}$ & & $\frac{-359560490231}{5809013758675431109757952000}$\\
	$x^{14}$ & & $\frac{77182092353629}{260609784255455865877071000576000}$
\end{longtable}
\begin{algorithm}[!ht]
	\caption[Bilinear perturbation of $\frac{\partial^2 e^X}{\partial x \partial y}$]{Numerical calculation of the bilinear perturbation of the Hessian of the matrix exponential $\frac{\partial^2 e^X}{\partial x \partial y}$.}
	\label{alg:bilinear}
	\begin{algorithmic}[1]
		\Function{BIL}{$X$,$D_x$,$D_y$,$\epsilon = \text{machine float precision}$}
			\State $A_x \gets \left[X,D_x\right]$\Comment{Allocates memory}
			\State $A_y \gets \left[X,D_y\right]$\Comment{Allocates memory}
			\If{$A_x = 0$ \textbf{and} $A_y = 0$}
				\State $Z \gets 0$\Comment{Commutative short circuit}
			\ElsIf{$A_x = 0$}			
				\State $A_X \gets I \otimes X - X^\dagger \otimes I$\Comment{if $X$ is $n \times n$ the result is $n^2 \times n^2$}
				\State $\vec{a_y} \gets \operatorname{VEC}\left(A_y\right)$\Comment{Change of indexing}
				\State $P \gets P_{12}\left(A_X\right)$\Comment{Recursive squaring and summing}
				\State $Q \gets Q_{14}\left(A_X\right)$\Comment{Recursive squaring and summing}
				\State Solve for $\vec{z}$: $P \vec{a_y} = Q \vec{z}$\Comment{Call to linear solver}
				\State $Z \gets \operatorname{MAT}\left(\vec{z}\right)$\Comment{Change of indexing}
				\State $Z \gets \left[D_x, Z\right]$\Comment{In place computation}
			\ElsIf{$A_y = 0$}
				\State $A_X \gets I \otimes X - X^\dagger \otimes I$\Comment{if $X$ is $n \times n$ the result is $n^2 \times n^2$}
				\State $\vec{a_x} \gets \operatorname{VEC}\left(A_x\right)$\Comment{Change of indexing}
				\State $P \gets P_{12}\left(A_X\right)$\Comment{Recursive squaring and summing}
				\State $Q \gets Q_{14}\left(A_X\right)$\Comment{Recursive squaring and summing}
				\State Solve for $\vec{z}$: $P \vec{a_x} = Q \vec{z}$\Comment{Call to linear solver}
				\State $Z \gets \operatorname{MAT}\left(\vec{z}\right)$\Comment{Change of indexing}
				\State $Z \gets \left[D_y, Z\right]$\Comment{In place computation}
			\Else
				\State $Z \gets 0$\Comment{Allocates memory}
				\State $F \gets \left[D_x, A_y\right] + \left[D_y,A_x\right]$\Comment{Allocates memory}
				\State $m \gets 3$\Comment{Factorial scalars}
				\State $n \gets 6$\Comment{Factorial scalars}
				\While{$ \left\| F \right\| > n\epsilon$}
					\State $Z \gets Z + \frac{F}{n}$\Comment{In place computation}
					\State $A_x \gets \left[X,A_x\right]$\Comment{In place computation}
					\State $A_y \gets \left[X,A_y\right]$\Comment{In place computation}
					\State $F \gets \left[D_x, A_y\right] + \left[X,F\right] + \left[D_x,A_x\right]$\Comment{In place computation}
					\State $m \gets m+1$\Comment{Increment factorial}
					\State $n \gets m n$\Comment{Increment factorial}
				\EndWhile
			\EndIf
			\State \Return $Z$
		\EndFunction
	\end{algorithmic}
\end{algorithm}
\begin{algorithm}[!ht]
	\caption[Hessian of $e^X$]{Numerical calculation of the Hessian of the matrix exponential $\frac{\partial^2 e^X}{\partial x \partial y}$.}
	\label{alg:hessian}
	\begin{algorithmic}[1]
		\Function{HEX}{$X$,$D_x$,$D_y$,$D_{xy}$}
			\State $E \gets e^X$\Comment{Call to matrix exponential}
			\State $Z \gets D_{xy} + \operatorname{PER}\left(X,D_{xy}\right)$\Comment{Call to non-commutative perturbation}
			\State $Z \gets Z + \frac{1}{2} \operatorname{BIL}\left(X,D_x,D_y\right)$\Comment{Call to bilinear perturbation}
			\State $P_x \gets D_x + \operatorname{PER}\left(X,D_x\right)$\Comment{Call to non-commutative perturbation}
			\State $P_y \gets D_y + \operatorname{PER}\left(X,D_y\right)$\Comment{Call to non-commutative perturbation}
			\State $Z \gets Z + \frac{1}{2}\left(P_x P_y + P_y P_x\right)$\Comment{In place computation}
			\State $Z \gets Z E$\Comment{In place computation}
			\State \Return $Z$
		\EndFunction
	\end{algorithmic}
\end{algorithm}