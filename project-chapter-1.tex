\chapter{Introduction}
\section{Motivation and Direction}
Markov process have a rich and extensive history in the statistical and 
probability analysis of fields such as the life sciences, operations research,
queuing theory communications, natural language, finance and machine learning.
Have modeled syntax of sentences, disease states, cancer survival, epidemiology
and demographics, with models such as phase type, birth-death, and hidden
State of the art computational methods are focused on hidden Markov Models on
a finite state set, and discrete time steps; assuming all transitions are
observed but are obscured with noise. This is the focus of the Baum-Welch and
Viterbi algorithms, and more recently RUST models. This is because Markov models
have an intuitive phenomenological interpretation.

The reason for the focus on discrete time finite state Markov processes is that
in dimension greater than four the matrix exponential does not have a closed form
entry wise. This is because polynomials of degree greater than four are not
generally solvable. Instead computational approximations for the matrix
exponential must be used. This work extends and generalizes those computational
methods to include the gradient and Hessian of the matrix exponential.

The work presented in this project concerns continuous time Markov processes on
a finite state space, where by definition it is not possible to observe all the
transitions. Instead what is observed typical is are stopped statistics, such
as first hitting times from one state to another.

The second chapter establishes the algebraic and analytic closure properties
necessary for chapter three. Chapter two has a secondary role to help develop the 
physical intuition for the stochastic Lie group necessary to work through the 
derivatives and approximations of chapter three.
The third chapter derives the the first and second order derivatives of the exponential
map and their Pad\'{e} Approximation.
The fourth chapter derives the maximum likelihood estimators from first hitting times
The fifth chapter concludes with summarizing remarks and a discussion of the direction
for further investigation.
Throughout this work we will attempt to conform to a simplified version of 
Lamport's guide to structuring and presenting proofs.
\section{Background}
\subsection{Continuous Time Markov Process}
\subsection{Maximum Likelihood Estimation}
\subsection{Lie Theory}
\subsection{Pad\'{e} Approximation}
\subsection{Newton-Raphson Method}
