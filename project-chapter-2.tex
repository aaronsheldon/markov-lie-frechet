\chapter{The Lie Algebra of the Generators of Continuous Time Markov Processes}
\section{Stochastic Matrices}
The classical Lie Algebras of physics, like the infinitesimal symmetries
of the special unitary algebra $\mathfrak{su}(n)$, are defined with respect to
invariants of a Banach algebra, such as the matrix invariants of the 
determinant, trace, or norm. In contrast stochastic matrices are always 
characterized with respect to a specific unit vector, which we will denote 
$\hat{\mathbbm{1}}$. In the next two subsections we provide an explicit 
construction and characterization building on the original the work of ??.

\begin{definition}
	A matrix $A$ is stochastic with respect to the unit vector $\hat{\mathbbm{1}}$ 
	if $A \hat{\mathbbm{1}} = \hat{\mathbbm{1}}$
\end{definition}

In an $n$ dimensional vector space the vector $\vec{\mathbbm{1}} = \sqrt{n} \hat{\mathbbm{1}}$ 
acts as the row sum operator on matrices stochastic with respect to $\hat{\mathbbm{1}}$. 
We will make this claim more precise after we dispense we a few more 
foundational definitions.

\begin{definition}
	Let $St(\hat{\mathbbm{1}})$ denote the group of invertible matrices 
	stochastic with respect to $\hat{\mathbbm{1}}$
\end{definition}

This definition immediately necessitates proof of the claim in the definition.

\begin{lemma}
	$St(\hat{\mathbbm{1}})$ is a Lie group
\end{lemma}

\begin{proof}
	We proceed by working mechanistically through the Lie group axioms.
	\begin{enumerate}
		\item The identity element $I$ is in $St(\hat{\mathbbm{1}})$. Clearly $I$ is
		invertible and $I \hat{\mathbbm{1}} = \hat{\mathbbm{1}}$.
		\item If $A,B \in St(\hat{\mathbbm{1}})$ then $AB \in St(\hat{\mathbbm{1}})$. 
		This follows from the computation $AB \hat{\mathbbm{1}} = A \hat{\mathbbm{1}} = \hat{\mathbbm{1}}$.
		\item If $A \in St(\hat{\mathbbm{1}})$ then $A^{-1} \in St(\hat{\mathbbm{1}})$.
		Recognize that $A \hat{\mathbbm{1}} = \hat{\mathbbm{1}}$ implies $\hat{\mathbbm{1}} = A^{-1} A \hat{\mathbbm{1}} = A^{-1} \hat{\mathbbm{1}}$.
		\item Associativity follows from $St(\hat{\mathbbm{1}})$ being a subgroup of $GL\left(n\right)$.
		\item Finally that the matrix product $A^{-1}B$ is smooth for all $A,B \in St(\hat{\mathbbm{1}})$
		likewise follows from $St(\hat{\mathbbm{1}}) < GL\left(n\right)$
	\end{enumerate}
\end{proof}

That $St(\hat{\mathbbm{1}})$ is a proper matrix Lie group implies that it must be
infinitesimal generated by elements of a Lie algebra.

\begin{definition}
	Let $\mathfrak{st}(\hat{\mathbbm{1}})$ denote the Lie algebra of $St(\hat{\mathbbm{1}})$
\end{definition}

We can immediately fully characterize this algebra as the set of matrices such
their row sums are zero with respect to $\hat{\mathbbm{1}}$.

\begin{lemma}
	The algebra $\mathfrak{st}(\hat{\mathbbm{1}})$ is exactly the set of all 
	matrices with $\hat{\mathbbm{1}}$ in their kernel.
\end{lemma}

\begin{proof}
	Working through the forward and backward inclusions we have
	\begin{enumerate}
		\item Suppose $A \hat{\mathbbm{1}} = 0$ then from the definition of the 
		matrix exponential we have:
		% IEEE equation box with matrix summation formula, strictly speaking this shows
		% that the domain of convergence of the series includes the unit vector.
		
		Thus $A \in \mathfrak{st}(\hat{\mathbbm{1}})$
		\item Now begin with the reverse assumption, that $A \in \mathfrak{st}(\hat{\mathbbm{1}})$.
		For all $t \in \mathbb{R}$ we have $\exp\left(tA\right) \hat{\mathbbm{1}}= \hat{\mathbbm{1}}$.
		Differentiation with respect to $t$ and evaluation at $t = 0$ yields $A \hat{\mathbbm{1}} = 0$
		
		% IEEE equation box with derivative
	\end{enumerate}
\end{proof}

Over an $n$ dimensional vector space, the condition on the matrix $A$ that $A \hat{\mathbbm{1}} = 0$
places $n$ constraints on the $n^2$ dimensions of $A$. This leaves $n^2 - n$ 
free dimensions on $\mathfrak{st}(\hat{\mathbbm{1}})$, when considered as a
vector space. This hints that we can construct a generator of $\mathfrak{st}(\hat{\mathbbm{1}})$
from order pairs of basis elements $\hat{e}_i$ for the vector space of $\hat{\mathbbm{1}}$.
To see how this is done we first construct a useful basis for the vector space
of $\hat{\mathbbm{1}}$.

\begin{lemma}
	There exists an orthonormal basis $\hat{e}_i$ such that $\left(\hat{e}_i, \hat{\mathbbm{1}} \right) = \frac{1}{\sqrt{n}}$
	for all $i$
\end{lemma}

\begin{proof}
	Invoke Gram-Schmidt orthogonalization
\end{proof}

We now can proceed with the central result that motivates this chapter.

\begin{theorem}
	The canonical generators of $\mathfrak{st}(\hat{\mathbbm{1}})$ are $C_{ij} = \frac{1}{\sqrt{2}} \hat{e}_i \otimes \left( \hat{e}_j - \hat{e}_i \right)$
\end{theorem}

\begin{proof}
	Show that the smallest containing algebra is the whole algebra.
\end{proof}

The previous theorem serves as the definition of the canonical generators of $\mathfrak{st}(\hat{\mathbbm{1}})$.
The first result from this theorem is the ability to calculate the structure
constants of the generators of the algebra. We proceed by studying the products
of the generators.

\begin{corollary}
	$C_{ij}C_{kl} = \delta$
\end{corollary}

\begin{corollary}
	$\exp\left(C_{ij}\right) = e$
\end{corollary}

% hmmm does the convex hull of the canonical generators form all the true 
% positive definite stochastic matrices, probably not, there doesn't seem to be
% enough of them.

\section{Doubly Stochastic Matrices}
Doubly stochastic matrices require double conservation of the vector $\hat{\mathbbm{1}}$, leaving
only $\left(n - 1\right)^2$ linear degrees of freedom. This is an important clue in the construction
of a canonical representation. In fact the representation can be found by choosing one additional
vector $\hat{e}_n$ to ``omit''. This vector plays a similar role to the diagonal in the previous
construction and is used to balance the row and column sums back to zero.
$St(\hat{\mathbbm{1}},\hat{\mathbbm{1}})$ and $\mathfrak{st}(\hat{\mathbbm{1}},\hat{\mathbbm{1}})$