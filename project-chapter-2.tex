\chapter{The Lie Algebra of the Generators of Continuous Time Markov Processes on Finite State Spaces}
\section{Stochastic Matrices}
The classical Lie algebras of physics, like the infinitesimal symmetries
of the special unitary algebra $\mathfrak{su}(n)$, are defined with respect to
invariants of a Banach algebra, such as the matrix invariants of the 
determinant, trace, or norm. In contrast stochastic matrices are always 
characterized with respect to a specific unit vector, which we will denote 
$\hat{\mathbbm{1}}$. In the next two sections we provide an explicit 
construction and characterization of the Lie algebra of stochastic matrices, 
building on the original the work of ??.

The common approach to stochastic matrices begins with the restriction that the 
matrices have non-negative entries with respect to the standard orthonormal 
basis for the vector space on which it acts; namely $\left\langle\hat{e}_i,A \hat{e}_j\right\rangle \ge 0$
for all $i,j$. In addition to allowing for singular matrices this poses an 
immediate obstacle to the necessary closure with respect to matrix inversion 
required for matrix groups. As the inverse of a stochastic matrix need not have 
non-negative entries with respect to the standard orthonormal basis. 

For the moment we will set aside the restriction that the entries be 
non-negative, and instead begin with a generalization of the concept of fixed 
row sums. We will show this generalization is preserved by matrix inversion, and 
then develop an orthonormal basis from which specific matrices with non-negative 
entries, with respect to the basis, can be constructed. In essence tackling the 
problem from the reverse direction, starting with the more general idea of fixed 
row sums, then specifying to matrices with non-negative entries with respect to 
a constructed orthonormal basis.

\begin{definition}
	A matrix $A$ is stochastic with respect to the unit vector $\hat{\mathbbm{1}}$ 
	if $A \hat{\mathbbm{1}} = \hat{\mathbbm{1}}$
\end{definition}

In an $n$ dimensional vector space the vector $\vec{\mathbbm{1}} = \sqrt{n} \hat{\mathbbm{1}}$ 
acts as the row sum operator on matrices stochastic with respect to $\hat{\mathbbm{1}}$. 
We will make this claim more precise after we dispense with a few more 
foundational definitions.

\begin{definition}
	Let $St(\hat{\mathbbm{1}})$ denote the stochastic Lie group of invertible 
	matrices stochastic with respect to $\hat{\mathbbm{1}}$
\end{definition}

It is tempting to view the name stochastic Lie group has a bait and switch, or 
at least an abuse of the terminology, given we have removed the usual convex
polytope of stochastic matrices and replaced it with a group of invertible 
matrices with a common eigenvector $\hat{\mathbbm{1}}$. Previous authors have 
denoted the convex polytope of stochastic matrices as the stochastic semi-group 
and the group invertible matrices as the pseudo-stochastic Lie group. One could 
even consider incorporating Markov into the name, in reference to the fact that 
the transition matrices of a continuous Markov process on a finite state space 
are by definition invertible and have common eigenvector $\hat{\mathbbm{1}}$. 
However the suffix of Lie group in the name connotes both sufficient additional 
restrictions to make the name distinct, and still allows for an indication of a 
relationship with the original concept. Of course, this definition immediately 
necessitates proof of the claim embedded in the definition.

\begin{lemma}
	$St(\hat{\mathbbm{1}})$ is a Lie group
\end{lemma}

\begin{proof}
	We proceed by working mechanistically through the Lie group axioms.
	\begin{enumerate}
		\item The identity element $I$ is in $St(\hat{\mathbbm{1}})$. Clearly $I$ is
		invertible and $I \hat{\mathbbm{1}} = \hat{\mathbbm{1}}$.
		\item If $A,B \in St(\hat{\mathbbm{1}})$ then $AB \in St(\hat{\mathbbm{1}})$. 
		This follows from the computation $AB \hat{\mathbbm{1}} = A \hat{\mathbbm{1}} = \hat{\mathbbm{1}}$.
		\item If $A \in St(\hat{\mathbbm{1}})$ then $A^{-1} \in St(\hat{\mathbbm{1}})$.
		Recognize that $A \hat{\mathbbm{1}} = \hat{\mathbbm{1}}$ implies $\hat{\mathbbm{1}} = A^{-1} A \hat{\mathbbm{1}} = A^{-1} \hat{\mathbbm{1}}$.
		\item Associativity follows from $St(\hat{\mathbbm{1}})$ being a subgroup of $GL\left(n\right)$.
		\item Finally, that the matrix product $A^{-1}B$ is smooth for all $A,B \in St(\hat{\mathbbm{1}})$
		likewise follows from $St(\hat{\mathbbm{1}}) < GL\left(n\right)$
	\end{enumerate}
\end{proof}

That $St(\hat{\mathbbm{1}})$ is a proper matrix Lie group implies that it must be
infinitesimal generated by elements of a Lie algebra.

\begin{definition}
	Let $\mathfrak{st}(\hat{\mathbbm{1}})$ denote the stochastic Lie algebra of $St(\hat{\mathbbm{1}})$
\end{definition}

By infinitesimally generated we mean that every element of $St(\hat{\mathbbm{1}})$
is a matrix exponential of some element in $\mathfrak{st}(\hat{\mathbbm{1}})$. 
We can fully characterize this algebra as the set of matrices such that their 
row sums are zero with respect to $\hat{\mathbbm{1}}$.

\begin{lemma}
	The algebra $\mathfrak{st}(\hat{\mathbbm{1}})$ is exactly the set of all 
	matrices with $\hat{\mathbbm{1}}$ in their kernel.
\end{lemma}

\begin{proof}
	Working through the forward and backward inclusions we have
	\begin{enumerate}
		\item Suppose $A \hat{\mathbbm{1}} = 0$ then from the definition of the 
		matrix exponential we have:
		\begin{IEEEeqnarray*}{rCl}
			\exp\left(A\right) \hat{\mathbbm{1}}
				& = & \sum_{n=0}^{\infty} \frac{1}{n!} A^n \hat{\mathbbm{1}}\\
				& = & \hat{\mathbbm{1}} + \sum_{n=1}^{\infty} \frac{1}{n!} 0\\
				& = & \hat{\mathbbm{1}}
		\end{IEEEeqnarray*}
		Thus $\exp\left(A\right) \in St(\hat{\mathbbm{1}})$ implying that $A \in \mathfrak{st}(\hat{\mathbbm{1}})$
		\item Now begin with the reverse assumption, that $A \in \mathfrak{st}(\hat{\mathbbm{1}})$.
		For all $t \in \mathbb{R}$ we have $\exp\left(tA\right) \hat{\mathbbm{1}} = \hat{\mathbbm{1}}$.
		Differentiation with respect to $t$ and evaluation at $t = 0$ yields
		\begin{IEEEeqnarray*}{rCl}
			0 & = & \left. \frac{d}{dt} \hat{\mathbbm{1}} \right|_{t=0}\\
				& = & \left. \frac{d}{dt} \exp\left(tA\right) \hat{\mathbbm{1}} \right|_{t=0}\\
				& = & \left. \exp\left(tA\right) A \hat{\mathbbm{1}} \right|_{t=0}\\
				& = & A \hat{\mathbbm{1}}
		\end{IEEEeqnarray*}
	\end{enumerate}
\end{proof}

Over an $n$ dimensional vector space, the condition on a matrix $A$ that $A \hat{\mathbbm{1}} = 0$
places $n$ constraints on the $n^2$ dimensions of $A$. This leaves $n^2 - n$ 
free dimensions on $\mathfrak{st}(\hat{\mathbbm{1}})$, when considered as a
vector space. This hints that we can construct a generator of $\mathfrak{st}(\hat{\mathbbm{1}})$
from order pairs of basis elements $\hat{e}_i$ for the vector space of $\hat{\mathbbm{1}}$.
To see how this is done we first construct a useful basis for the vector space
to which $\hat{\mathbbm{1}}$ is a member.

\begin{lemma}
	There exists an orthonormal basis $\hat{e}_i$ such that $\left\langle \hat{e}_i, \hat{\mathbbm{1}} \right\rangle = \frac{1}{\sqrt{n}}$
	for all $i$
\end{lemma}

\begin{proof}
	While a basis with the stipulated properties can be constructed through the
	Gram-Schmidt process, the proof of the existence proceeds by induction.
	\begin{enumerate}
		\item For $n=1$ the desired basis is precisely the trivial set $\left\lbrace \hat{\mathbbm{1}} \right\rbrace$ 
		which satisfies the condition that $\left\langle \hat{\mathbbm{1}}, \hat{\mathbbm{1}} \right\rangle = 1$.
		\item Assume the claim is true for $n$. For $n+1$ pick a unit vector $\hat{e}_{\perp}$
		that is orthogonal to $\hat{\mathbbm{1}}$ and construct the unit vector
		$\hat{e}_{n+1} = \frac{1}{\sqrt{n+1}} \hat{\mathbbm{1}} + \sqrt{\frac{n}{n+1}} \hat{e}_{\perp}$.
		Clearly $\hat{e}_{n+1}$ satisfies the condition $\left\langle \hat{e}_{n+1}, \hat{\mathbbm{1}} \right\rangle = \frac{1}{\sqrt{n+1}}$.
		\item To use the the induction assumption we construct a new row sum unit 
		vector $\hat{\mathbbm{1}}_n = \sqrt{\frac{n+1}{n}}\hat{\mathbbm{1}} - \frac{1}{\sqrt{n}}\hat{e}_{n+1}$ 
		in one dimension lower by projecting onto the subspace orthogonal to $\hat{e}_{n+1}$.
		\item By the induction assumption there exists a basis $\hat{e}_i$ with $i \le n$, 
		such that $\left\langle \hat{e}_i, \hat{\mathbbm{1}}_n \right\rangle = \frac{1}{\sqrt{n}}$.
		\item Because $\hat{e}_i$ with $i \le n$ was constructed in the space 
		orthogonal to $\hat{e}_{n+1}$ if follows that $\left\langle \hat{e}_i, \hat{e}_j \right\rangle = \delta_{ij}$
		for all $i,j \le n+1 $.
		\item Then using the definitions of $\hat{e}_{n+1}$ and $\hat{\mathbbm{1}}_n$
		we can calculate the inner product $\left\langle \hat{e}_i, \hat{\mathbbm{1}}_n \right\rangle$
		for $i \le n$
		\begin{IEEEeqnarray*}{rCl}
			\frac{1}{\sqrt{n}}
				& = & \left\langle \hat{e}_i, \hat{\mathbbm{1}}_n \right\rangle\\
				& = & \sqrt{\frac{n+1}{n}} \left\langle \hat{e}_j, \hat{\mathbbm{1}} \right\rangle - \frac{1}{\sqrt{n}} \left\langle \hat{e}_i, \hat{e}_{n+1} \right\rangle\\
				& = & \sqrt{\frac{n+1}{n}} \left\langle \hat{e}_j, \hat{\mathbbm{1}} \right\rangle
		\end{IEEEeqnarray*}
		Inverting the fraction in the equality yields $\left\langle \hat{e}_i, \hat{\mathbbm{1}} \right\rangle = \frac{1}{\sqrt{n+1}}$
		for all $i \le n+1$.
	\end{enumerate}
\end{proof}

As a direct result of the construction of the basis vectors $\hat{e}_i$ we see 
that $\vec{\mathbbm{1}} = \sum_{i=1}^n \hat{e}_i$. Thus $\vec{\mathbbm{1}}$ can
be interpreted as the row sum vector in basis $\hat{e}_i$.

The constructed basis leads naturally to considering the minimal non-trivial 
matrices $C_{ij} = \hat{e}_i \otimes \left( \hat{e}_j - \hat{e}_i \right)$ as 
holding significance in the structure of $\mathfrak{st}(\hat{\mathbbm{1}})$.
In fact this will be the central result of this chapter: that the algebraic 
closure of the matrices $C_{ij}$ is the stochastic Lie algebra $\mathfrak{st}(\hat{\mathbbm{1}})$. 
To establish this result we need a preliminary result that proves the 
commutators $\left[C_{ij},C_{kl}\right]$ are linear combinations of matrices $C_{ij}$.

\begin{lemma}
	\begin{IEEEeqnarray*}{rCl}
		C_{ij}C_{kl} & = &
		\begin{cases}
			- C_{il} & i=k,\\
			C_{il} - C_{ij} & j=k,\\
			0 & \text{otherwise}.
		\end{cases}
	\end{IEEEeqnarray*}
\end{lemma}

\begin{proof}
	We proceed in two steps; calculating the terms of the products, then 
	simplifying the cases, always assuming $i \neq j$ and $k \neq l$.
	\begin{enumerate}
		\item Term wise computation yields
		\begin{IEEEeqnarray*}{rCl}
			C_{ij}C_{kl}
				& = & \hat{e}_i \otimes \left( \hat{e}_j - \hat{e}_i \right) \hat{e}_k \otimes \left( \hat{e}_l - \hat{e}_k \right)\\
				& = & \hat{e}_i \otimes \hat{e}_j \hat{e}_k \otimes \hat{e}_l + \hat{e}_i \otimes \hat{e}_i \hat{e}_k \otimes \hat{e}_k - \hat{e}_i \otimes \hat{e}_j \hat{e}_k \otimes \hat{e}_k - \hat{e}_i \otimes \hat{e}_i \hat{e}_k \otimes \hat{e}_l\\
				& = & \delta_{jk} \hat{e}_i \otimes \hat{e}_l + \delta_{ik} \hat{e}_i \otimes \hat{e}_k - \delta_{jk} \hat{e}_i \otimes \hat{e}_k - \delta_{ik} \hat{e}_i \otimes \hat{e}_l\\
				& = & \left(\delta_{jk} - \delta_{ik} \right) \hat{e}_i \otimes \left( \hat{e}_l - \hat{e}_k \right)
		\end{IEEEeqnarray*}
		\item We work through each case of $\delta_{jk} - \delta_{ik}$, starting 
		with the case $i=k$ 
		\begin{IEEEeqnarray*}{rCl}
			C_{ij}C_{il}
				& = & \left(\delta_{jk} - \delta_{ii} \right) \hat{e}_i \otimes \left( \hat{e}_l - \hat{e}_i \right)\\
				& = & - \hat{e}_i \otimes \left( \hat{e}_l - \hat{e}_i \right)\\
				& = & - C_{il}
		\end{IEEEeqnarray*}
		\item When $j=k$ we have
		\begin{IEEEeqnarray*}{rCl}
			C_{ij}C_{jl}
				& = & \left(\delta_{jj} - \delta_{ij} \right) \hat{e}_i \otimes \left( \hat{e}_l - \hat{e}_j \right)\\
				& = & \hat{e}_i \otimes \left( \hat{e}_l - \hat{e}_j \right)\\
				& = & \hat{e}_i \otimes \left( \hat{e_l} - \hat{e}_i + \hat{e}_i - \hat{e}_j \right)\\
				& = & C_{il} - C_{ij}
		\end{IEEEeqnarray*}
		\item Finally when none of the previous conditions apply
		\begin{IEEEeqnarray*}{rCl}
			C_{ij}C_{kl}
				& = & \left(\delta_{jk} - \delta_{ik} \right) \hat{e}_i \otimes \left( \hat{e}_l - \hat{e}_k \right)\\
				& = & 0 \cdot \hat{e}_i \otimes \left( \hat{e}_l - \hat{e}_k \right)\\
				& = & 0
		\end{IEEEeqnarray*}
	\end{enumerate}
\end{proof}

While this result is sufficient to accomplish the central result, it is worth
carrying through with the computation of the structure constants of the generators.

\begin{corollary}
	\begin{IEEEeqnarray*}{rCl}
		\left[C_{ij},C_{kl}\right] & = &
		\begin{cases}
			C_{ij} - C_{il} & i=k,\\
			C_{kj} - C_{ki} & i=l,\\
			C_{il} - C_{ij} & j=k,\\
			0 & \text{otherwise}.
		\end{cases}
	\end{IEEEeqnarray*}
\end{corollary}

\begin{proof}
	As in the previous lemma we work case wise through the equalities.
	\begin{enumerate}
		\item Starting with $i=k$
		\begin{IEEEeqnarray*}{rCl}
			\left[C_{ij},C_{il}\right]
				& = & C_{ij}C_{il} - C_{il}C_{ij}\\
				& = & C_{ij} - C_{il}
		\end{IEEEeqnarray*}
		\item For $i=l$
		\begin{IEEEeqnarray*}{rCl}
			\left[C_{ij},C_{ki}\right]
				& = & C_{ij}C_{ki} - C_{ki}C_{ij}\\
				& = & C_{kj} - C_{ki}
		\end{IEEEeqnarray*}
		\item For $j=k$
		\begin{IEEEeqnarray*}{rCl}
			\left[C_{ij},C_{jl}\right]
				& = & C_{ij}C_{jl} - C_{jl}C_{ij}\\
				& = & C_{il} - C_{ij}
		\end{IEEEeqnarray*}
		\item When none of the conditions apply
		\begin{IEEEeqnarray*}{rCl}
			\left[C_{ij},C_{kl}\right]
				& = & C_{ij}C_{kl} - C_{kl}C_{ij}\\
				& = & 0
		\end{IEEEeqnarray*}
	\end{enumerate}
\end{proof}

We can now proceed with the central result that motivates this chapter.

\begin{theorem}
	The canonical generators of $\mathfrak{st}(\hat{\mathbbm{1}})$ are $C_{ij}$
\end{theorem}

\begin{proof}
	The previous lemma has established that the products, and thus the 
	commutators, of $C_{ij}$ are linear in $C_{ij}$. We then have to prove that 
	the smallest algebra that contains $C_{ij}$ is $\mathfrak{st}(\hat{\mathbbm{1}})$. 
	As thus, it is sufficient to prove that matrices $C_{ij}$ from a basis for $\mathfrak{st}(\hat{\mathbbm{1}})$.
	This is because a necessary condition for an algebra to contain the matrices
	$C_{ij}$ is that it must contain all sums of the matrices $C_{ij}$. If one
	could sum their way out of the algebra then it would not be an algebra.
	\begin{enumerate}
		\item That $\mathfrak{st}(\hat{\mathbbm{1}})$ is an $n^2-n$ dimensional 
		vector space should be clear from the previous discussion. A full formal
		proof of this claim is found through induction on the dimension $n$.
		\item The matrices $C_{ij}$ are in $\mathfrak{st}(\hat{\mathbbm{1}})$. From
		the definition of the canonical generators
		\begin{IEEEeqnarray*}{rCl}
			C_{ij} \hat{\mathbbm{1}}
				& = & \hat{e}_i \otimes \left( \hat{e}_j - \hat{e}_i \right) \hat{\mathbbm{1}}\\
				& = & \hat{e}_i \left( \left\langle \hat{e}_j, \hat{\mathbbm{1}} \right\rangle - \left\langle \hat{e}_i, \hat{\mathbbm{1}} \right\rangle \right)\\
				& = & \hat{e}_i \left(\frac{1}{\sqrt{n}} - \frac{1}{\sqrt{n}}\right)\\
				& = & 0
		\end{IEEEeqnarray*}
		\item $C_{ij}$ is a set of $n^2-n$ linear independent matrices and so must
		form a basis for all of $\mathfrak{st}(\hat{\mathbbm{1}})$. That there are
		only $n^2-n$ matrices is clear from the fact that $C_{ii} = 0$. While the 
		formal proof of linear independence is again found through induction on the 
		dimension $n$.
	\end{enumerate}
\end{proof}

The previous theorem serves as the definition of a set of canonical generators 
of $\mathfrak{st}(\hat{\mathbbm{1}})$. It is  important to note that neither the 
basis $\hat{e}_i$ nor the canonical generators $C_{ij}$ are unique. They are 
uniquely defined only up to rotations orthogonal to the vector $\hat{\mathbbm{1}}$.

Since $C_{ij} \in \mathfrak{st}(\hat{\mathbbm{1}})$ its matrix exponential must
be in $St(\hat{\mathbbm{1}})$, which can be summarized in the following 
corollary.

\begin{corollary}
	$\exp\left(\alpha C_{ij}\right) = I + \left(1 - e^{-\alpha} \right)C_{ij}$
\end{corollary}

\begin{proof}
	From the definition of the matrix exponential
	\begin{IEEEeqnarray*}{rCl}
		\exp\left(\alpha C_{ij}\right)
			& = & \sum_{n=0}^{\infty} \frac{1}{n!}\alpha^n C_{ij}^n\\
			& = & I + \sum_{n=1}^{\infty} \frac{1}{n!} \left(-1\right)^{n+1} \alpha^n C_{ij}\\
			& = & I + \left(1 - e^{-\alpha} \right)C_{ij}
	\end{IEEEeqnarray*}	
\end{proof}

This last corollary admits a intuitive heuristic interpretation: that each 
canonical generator $C_{ij}$ can be thought of as measuring the infinitesimal 
transition rate, or flow of probability, from the state represented by basis 
element $\hat{e}_i$ to the state represented by the basis element $\hat{e}_j$. 
This can be seen by considering the matrix representation of $\exp\left(\alpha C_{ij}\right)$ 
in the basis spanned by $\hat{e}_i$ and $\hat{e}_j$.

\begin{IEEEeqnarray*}{rCl}
	I + \left(1 - e^{-\alpha} \right)C_{ij}
		& = & 
		\begin{pmatrix}
			1 & 0\\
			0 & 1
		\end{pmatrix}
		+ \left(1 - e^{-\alpha} \right)
		\begin{pmatrix}
			-1 & 1\\
			0 & 0
		\end{pmatrix}\\
		& = &
		\begin{pmatrix}
			e^{-\alpha} & 1 - e^{-\alpha}\\
			0 & 1
		\end{pmatrix}
\end{IEEEeqnarray*}

Despite the fact that $\mathfrak{st}(\hat{\mathbbm{1}})$ is a real vector space
the limits in the positive directions of $C_{ij}$ are finite, namely in the
basis spanned by $\hat{e}_i$ and $\hat{e}_j$

\begin{IEEEeqnarray*}{rCl}
	\lim_{\alpha \rightarrow \infty} \exp\left(\alpha C_{ij} \right)
		& = & I + C_{ij}\\
		& = &
		\begin{pmatrix}
			0 & 1\\
			0 & 1
		\end{pmatrix}
\end{IEEEeqnarray*}

% Restate next pargraph as general property of matrix Lie groups and algebras

For a general element of $\mathfrak{st}(\hat{\mathbbm{1}})$, say, $G = \sum_{ij} g_{ij} C_{ij}$ 
the matrix exponential $\exp\left(G\right)-I$ is by definition an element of $\mathfrak{st}(\hat{\mathbbm{1}})$.
Which because $\mathfrak{st}(\hat{\mathbbm{1}})$ is a vector space immediately 
implies that $\exp\left(G\right) = I + \sum_{ij} h_{ij} C_{ij}$. Unfortunately, 
because polynomials of degree greater than two are generally unsolvable, the 
relationship between the coefficients $g_{ij}$ and $h_{ij}$ is highly 
non-trivial in most circumstances.

At this point it is worth briefly revisiting the distinction between the 
standard convex polytope of stochastic matrices and the stochastic Lie group, to
develop some physical intuition into the relationship between the two sets of
matrices which have a non-trivial and geometrical interesting intersection.

% need to keep going, look at n=2,3

The vertexes of the convex polytope of stochastic matrices are exactly the $n^n$
matrices of the form

\begin{IEEEeqnarray*}{rCl}
	V_k 
		& = & \sum_{i=1}^n \hat{e}_i \otimes \hat{e}_{j_n^i\left(k\right)}\\
		& = & I + \sum_{i=1}^n C_{i j_n^i\left(k\right)}\\
		& = & \left(1 - n\right) I + \sum_{i=1}^n I + C_{i j_n^i\left(k\right)}\\
		& = & \left(1 - n\right) I + \sum_{i=1}^n \lim_{\alpha \rightarrow \infty} \exp\left(\alpha C_{i j_n^i\left(k\right)}\right)
\end{IEEEeqnarray*}

where $j_n^i\left(k\right) = 1 + \left\lfloor \frac{k}{n^{i-1}} \right\rfloor \mod n$ 
is $1$ plus the $i$ digit of $k \le n^n$ in base $n$.

The basis elements $\hat{e}_i$ are a well defined enumeration the states of a 
continuous Markov process on a finite state space; in that when the multipliers 
of the canonical generators are non-negative the matrix exponential gives a 
proper transition matrix for the process. The reverse is also true, up to choice
of branch of the matrix logarithm, and can be summarized in the following 
corollary.

\begin{corollary}
	$G = \sum_{ij} g_{ij} C_{ij}$ is the generator of a continuous time Markov 
	process on a finite state space enumerated by $\hat{e}_i$ if and only if $g_{ij} \ge 0$
\end{corollary}

We have developed an interpretation of the Eigen equation $A \hat{\mathbbm{1}} = \hat{\mathbbm{1}}$
as a conservation of the row sums of $A$; likewise the Eigen equation $A^T \hat{\mathbbm{1}} = \hat{\mathbbm{1}}$
can be interpreted as the conservation of the column sums of $A$. The dual 
definitions for the Lie group and algebra follow natural.

\begin{definition}
	Let $St^T(\hat{\mathbbm{1}})$ denote the dual stochastic Lie group of 
	invertible matrices whose transpose is stochastic with respect to $\hat{\mathbbm{1}}$.
\end{definition}

\begin{definition}
	Let $\mathfrak{st}^T(\hat{\mathbbm{1}})$ denote the dual stochastic Lie 
	algebra of $St^T(\hat{\mathbbm{1}})$.
\end{definition}

Thus if $C_{ij}$ are generators of $\mathfrak{st}(\hat{\mathbbm{1}})$ then $C_{ij}^T = \left(\hat{e}_j - \hat{e}_i \right) \otimes \hat{e}_i$
are generators of $\mathfrak{st}^T(\hat{\mathbbm{1}})$. That $St(\hat{\mathbbm{1}}) \cap St^T(\hat{\mathbbm{1}})$ 
is a Lie group and $\mathfrak{st}(\hat{\mathbbm{1}}) \cap \mathfrak{st}^T(\hat{\mathbbm{1}})$ 
is a Lie algebra will be used extensively in the next section.

\section{Doubly Stochastic Matrices}
Doubly stochastic matrices require row and column conservation of the vector $\hat{\mathbbm{1}}$, 
in the sense that both $A \hat{\mathbbm{1}} = \hat{\mathbbm{1}}$ and $A^T \hat{\mathbbm{1}} = \hat{\mathbbm{1}}$ 
must hold. This leaves only $\left(n - 1\right)^2$ linear degrees of freedom. 
This is an important clue in the construction of a canonical representation. In 
fact the representation can be found by choosing one additional vector $\hat{e}_n$ 
to center the combinatorial construction of the generators of the algebra 
around. This vector plays a similar role to the diagonal in the previous 
construction and is used to balance the row and column sums back to zero; as 
such we start with the definition

\begin{definition}
	Let $St(\hat{\mathbbm{1}}, \hat{\mathbbm{1}})$ denote the doubly stochastic 
	Lie group of invertible matrices $A$ such that both $A$ and $A^T$ are 
	stochastic with respect to $\hat{\mathbbm{1}}$
\end{definition}

We can immediately observe with out proof that $St(\hat{\mathbbm{1}}, \hat{\mathbbm{1}}) = St(\hat{\mathbbm{1}}) \cap St^T(\hat{\mathbbm{1}})$;
leading to the next definition.

\begin{definition}
	Let $\mathfrak{st}(\hat{\mathbbm{1}}, \hat{\mathbbm{1}})$ denote the doubly 
	stochastic Lie algebra of $St(\hat{\mathbbm{1}})$.
\end{definition}

Again, it should be clear that $\mathfrak{st}(\hat{\mathbbm{1}}, \hat{\mathbbm{1}}) = \mathfrak{st}(\hat{\mathbbm{1}}) \cap \mathfrak{st}^T(\hat{\mathbbm{1}})$.
The implication being that $\mathfrak{st}(\hat{\mathbbm{1}}, \hat{\mathbbm{1}})$
is the algebra of all matrices $A$ such that $\hat{\mathbbm{1}}$ is in the 
kernel of both $A$ and $A^T$. We can then find canonical generators of $\mathfrak{st}(\hat{\mathbbm{1}}, \hat{\mathbbm{1}})$

% preface this with products, commutators, and structure constants

\begin{theorem}
	$C_{ijn} = C_{ij} - C_{in} - C_{nj}$ are canonical generators of $\mathfrak{st}(\hat{\mathbbm{1}}, \hat{\mathbbm{1}})$.
\end{theorem}

\begin{proof}
	The proof proceeds in the same manner as the proof of the sibling theorem in
	the previous section.
\end{proof}

Sensibly, the canonical generators are closed with respect to transpositions, 
since $C_{ijn}^T = C_{jin}$. As with the stochastic Lie algebra the generators 
of the doubly stochastic Lie algebra are not unique, not only do they depend on 
the choice of the basis $\hat{e}_i$ but also on the choice of the basis element 
$\hat{e}_n$ used to sum the rows and columns to zero.

% Discuss Birkhoff polytope versus Lie group, construct vertexes vis-a-vi 
% Birkhoff-von Neumann theorem. Again vector space structure means every
% DS is linear sum.