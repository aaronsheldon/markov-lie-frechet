\chapter{Lie Algebra of the Generators of Continuous Time Markov Processes}
\section{Canonical Representation of the Generators of a Markov Process}
\subsection{Stochastic Matrices}
The classical Lie Algebras of physics, like the infinitesimal symmetries
of the special unitary algebra $\mathfrak{su}(n)$, can all be defined in terms
of invariants of a Banach space, such as the matrix invariants of the 
determinant, trace, or norm. In contrast stochastic matrices are always 
characterized with respect to a chosen vector, which we will denote 
$\hat{\mathbbm{1}}$, for reasons that will become clear later.
$ST(\hat{e})$ and $\mathfrak{st}(\hat{e})$
\subsection{Doubly Stochastic Matrices}
$ST(\hat{e},\hat{e})$ and $\mathfrak{st}(\hat{e},\hat{e})$
\section{Structure Constants of the Generator Algebra}
\subsection{Stochastic Matrices}
\subsection{Doubly Stochastic Matrices}