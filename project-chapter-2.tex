\chapter{Lie Algebra of the Generators of Continuous Time Markov Processes}
\section{Canonical Representation of the Generators of a Markov Process}
\subsection{Stochastic Matrices}
The classical Lie Algebras of physics, like the infinitesimal symmetries
of the special unitary algebra $\mathfrak{su}(n)$, can all be defined in terms
of invariants of a Banach space, such as the matrix invariants of the 
determinant, trace, or norm. In contrast stochastic matrices are always 
characterized with respect to a chosen vector, which we will denote 
$\hat{\mathbbm{1}}$, for reasons that will become clear later. For an $n \times n$
matrix this serves as $n$ linear constraints, leaving $n^2 - n$ linear degrees of
freedom.
\begin{definition}
	Let $St(\hat{\mathbbm{1}})$ denote the group of invertible matrices stochastic with respect to $\hat{\mathbbm{1}}$
\end{definition}
\begin{lemma}
	$St(\hat{\mathbbm{1}})$ is a Lie group
\end{lemma}
\begin{definition}
	Let $\mathfrak{st}(\hat{\mathbbm{1}})$ denote the Lie algebra of $St(\hat{\mathbbm{1}})$
\end{definition}
\begin{lemma}
	The canonical generators of $\mathfrak{st}(\hat{\mathbbm{1}})$ are $C_{ij} = \frac{1}{\sqrt{2}} \hat{e}_i \otimes \left( \hat{e}_j - \hat{e}_i \right)$
\end{lemma}
The previous lemma serves as the definition of the canonical generators of $\mathfrak{st}(\hat{\mathbbm{1}})$.
\begin{lemma}
	$\exp\left(C_{ij}\right) = e$
\end{lemma}
\begin{lemma}
	$C_{ij}C_{kl} = \delta$
\end{lemma}
\subsection{Doubly Stochastic Matrices}
Doubly stochastic matrices require double conservation of the vector $\hat{\mathbbm{1}}$, leaving
only $\left(n - 1\right)^2$ linear degrees of freedom. This is an important clue in the construction
of a canonical representation. In fact the representation can be found by choosing one additional
vector $\hat{e}_n$ to ``omit''. This vector plays a similar role to the diagonal in the previous
construction and is used to balance the row and column sums back to zero.
$St(\hat{\mathbbm{1}},\hat{e}_n)$ and $\mathfrak{st}(\hat{\mathbbm{1}},\hat{e}_n)$
\section{Structure Constants of the Generator Algebra}
\subsection{Stochastic Matrices}
\subsection{Doubly Stochastic Matrices}