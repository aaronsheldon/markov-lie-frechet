\chapter{Maximum Likelihood Estimation from First Hitting Times}
\section{Distribution of First Hitting Times}
Contemporary methods for fitting time homogeneous Markov processes on a finite 
state space require directly parameterizing the transition probability matrix $\mathbb{P}\left[X_n = j \left\|X_0 = i \right.\right] = \left\langle \hat{e}_i, P^n \hat{e}_j \right\rangle$,
as they depend on realizing the process through discrete time steps $n$. While this 
formulation has many powerful applications, there are analyses where the parameterization of 
the generator of the time homogeneous Markov process, $\mathbb{P}\left[X_t = j \left\|X_0 = i \right.\right] = \left\langle \hat{e}_i, \exp\left({tG}\right) \hat{e}_j \right\rangle$, 
is of greater meaning, or importance. In particular when the observed process is, at least
in principle continuous, or when the desired parameterization is in units of rates per time
parameterization of the generator is the more natural choice.

The natural experimental design for continuous time homogeneous Markov process on a finite
state space is the observation of a stopped process, such as the statistic of first hitting 
time of a state, or the first exit time from a state statistic. Surprisingly it is possible
to explicitly formulate the distribution of these two statistics in terms of the generator
of the process.

To start, recall the definition of the projection operator on a finite dimensional vector
space $P_i = \hat{e}_i \otimes \hat{e}_i$; which projects each vector in the space onto a
fixed unit vector $\hat{e}_i$. Projection operators hold a special purpose in analyzing
continuous time homogeneous Markov processes on a finite state spaces. The projection
operator $I-P_k$, when left multiplied to the generator $G = \sum_{ij}x_{ij}C_{ij}$, yields 
a new process $\left(I-P_k\right)G$, where state $k$ is an absorbing state.

The project operators $P_i$ are useful in formulating the distribution of first hitting 
times of a process generated by $G$ in terms of the transition probabilities of a process
generated by $\left(I-P_i\right)G$. We can apply this to restate the first hitting time 
results in the exercises of Rogers and Williams \cite{rogers_diffusions_2000}.
\begin{theorem}
	If $T_j = \inf\left\lbrace t: X_t=j\right\rbrace$ is the first 
	hitting time statistic of the transition to $j$ of a process generated by $G = \sum_{ij}x_{ij}C_{ij}$
	then
	\begin{IEEEeqnarray*}{rCl}
		\mathbb{P}_G\left[T_j \le t \left\| X_0=i \right.\right]
			& = & 1 - \left\langle \hat{e}_i, \exp\left(t\left(I - P_j\right) G\right) \hat{e}_j \right\rangle
	\end{IEEEeqnarray*}
\end{theorem}
\begin{IEEEproof}
	The proof hinges on formalizing the intuition that once we know a continuous time 
	homogeneous Markov process on a finite state space $X_t$ has first touched the state $j$ 
	then we need no further information, and so we can work with the simpler process where $j$
	is an absorbing state.
	\begin{IEEEeqnarray*}{+rCl+x*}
		\mathbb{P}_G\left[T_j \le t \left\| X_0=i \right.\right]
			& = & 1 - \mathbb{P}_G\left[T_j > t \left\| X_0=i \right.\right]\\
			& = & 1 - \mathbb{P}_G\left[\forall s \le t \enskip X_s \ne j, \enskip \exists u > t \enskip X_u=j \left\| X_0=i \right.\right]\\
			& = & 1 - \mathbb{P}_G\left[\forall s \le t \enskip X_s \ne j \left\| X_0=i \right.\right]\mathbb{P}_G\left[\exists u > t \enskip X_u=j \left\| \forall s \le t \enskip X_s \ne j \right.\right]\\
			& = & 1 - \mathbb{P}_G\left[\forall s \le t \enskip X_s \ne j \left\| X_0=i \right.\right]\mathbb{P}_G\left[X_u = j, u > t \left\| X_t \ne j \right.\right]\\
			& = & 1 - \mathbb{P}_{\left(I - P_j\right)G}\left[\forall s \le t \enskip X_s \ne j \left\| X_0=i \right.\right]\mathbb{P}_{\left(I - P_j\right)G}\left[X_u = j, \enskip u > t \left\| X_t \ne j \right.\right]\\
			& = & 1 - \mathbb{P}_{\left(I - P_j\right)G}\left[\forall s \le t \enskip X_s \ne j, \enskip X_u = j, \enskip u > t  \left\| X_0=i \right.\right]\\
			& = & 1 - \mathbb{P}_{\left(I - P_j\right)G}\left[X_t = j \left\| X_0=i \right.\right]\\
			& = & 1 - \left\langle \hat{e}_i, \exp\left(t\left(I - P_j\right) G\right) \hat{e}_j \right\rangle & \IEEEQEDhere
	\end{IEEEeqnarray*}
\end{IEEEproof}
This result generalizes in the obvious manner; where if we have a set of first hitting 
states $J$ then the transition probabilities of the process $\left(I - P_J\right) G$ gives
the cumulative distributions of the hitting times; where $P_J = P_{j_1} + P_{j_2} + \cdots$.
This implies that if we can design our experiment to observe as many of the first hitting 
times as possible we will greatly simplify our statistical estimators.

In light of this, we can reformulate the standard textbook result, for example in Buchholz 
et. al \cite{buchholz_input_2014}, of first hitting time statistics in the context of the
stochastic contraction Lie algebra $\mathfrak{st}^{+}(\hat{\mathbbm{1}})$. As is standard we 
start with an the experiment designed to observe the first exit time $T_{i \rightarrow j}$ 
from $\hat{e_i}$ to every other state $\hat{e}_j$, where $i \ne j$. Assuming the process is 
generated by $G = \sum_{ij}x_{ij}C_{ij}$, and keeping $i \ne j$ fixed, the density of the 
distribution of $T_{i \rightarrow j}$ is
\begin{IEEEeqnarray*}{rCl}
	p_G\left(T_{i \rightarrow j}=t \left\| X_0=i \right.\right)
		& = & \frac{d}{dt} \mathbb{P}_G\left[ T_j\le t \left\| X_0=i \right.\right]\\
		& = & \frac{d}{dt} \mathbb{P}_{P_i G}\left[ T_j\le t \left\| X_0=i \right.\right]\\
		& = & \frac{d}{dt} \left(1 - \left\langle \hat{e}_i, \exp\left(tP_iG\right) \hat{e}_j\right) \right\rangle\\
		& = & \frac{d}{dt} \left(1 - \left\langle \hat{e}_i, \exp\left(t\sum_{l \ne i} x_{il} C_{il}\right) \hat{e}_j\right) \right\rangle\\
		& = & \frac{d}{dt} \left(1 - \left\langle \hat{e}_i, e^{-tx_{ij}} \hat{e}_i\right) \right\rangle\\
		& = & \frac{d}{dt} \left(1 - e^{-tx_{ij}}\right)\\
		& = & x_{ij} e^{-tx_{ij}}
\end{IEEEeqnarray*}
Intuitively if we design our experiment to observer the durations $t_n = T_{i \rightarrow j}$
between $N_{ij}$ replicated transitions $i \rightarrow j$, the maximum likelihood estimate of
each rate $x_{ij}$ is then the simple average
\begin{IEEEeqnarray*}{rCl}
	\tilde{x}_{ij}
		& = & \frac{N_{ij}}{\sum_{n=1}^{N_{ij}} t_n}
\end{IEEEeqnarray*}
However for experiments that involve opportunistic sampling, surveys, or population
monitoring it is generally not possible to observe every distinct transition. Typically the
initial state of the transition is known or can be inferred, but only a subset of exit
states are observed. In this situation the projection operator $P_i$ onto a single
dimensional subspace is replaced with a projection $I - P_A = P_{i_1} + P_{i_2} + \cdots$ 
onto a multidimensional subspace; where $P_A$ is the projection onto the observed absorbing 
states in set $A$.
\section{The Likelihood and Its Maximization}
With a method to derive the density in hand we can proceed to formulate the log-likelihood
of the first hitting times. To do so we must carefully formulate the experimental design to 
which the log-likelihood will apply. Rather than attempt to formulate the most general
likelihood model possible, which would be notationally laborious given the infinite
permutations and combinations of models available, we will illustrate the formulation of
the likelihood through a specific application to an aging process.

An aging process is a continuous time homogeneous finite birth death-process, where all the
sequential transitions between states are reversible except for transitions to the final
state, which is an absorbing state representing death. In the context first hitting time
statistics, a finite subset of the states act as sentinel states, where the first hitting
time statistic for the transition between any pair of, possible non-adjacent, sentinel 
states is observed. An example of this process is illustrated in figure \ref{fig:agingprocess},
which displays a seven state aging process, with three sentinel states. The transitions
between states that are not sentinel are not directly observed; but rather acts as a type
of memory register that broadens the centrality of the distribution of first hitting times.

Given an $U$ state aging process, the generator takes on the simple sequential form:
\begin{IEEEeqnarray*}{rCl}
	G 
		& = & \sum_{i=1}^{U-1} x_{i\left(i+1\right)}C_{n\left(i+1\right)}\\
		& = & \sum_{i=1}^{U-1} x_{i\left(i+1\right)}\left(\hat{e}_{i} \otimes \hat{e}_{i+1} - \hat{e}_i \otimes \hat{e}_i\right)
\end{IEEEeqnarray*}
The $U$ state aging process as $2U-3$ unknown parameters, $x_{i\left( i+1\right)}$ for $1 \le i \le U-1$
and $x_{i\left( i-1\right)}$ for $2 \le i \le U-1$, to be fitted by likelihood maximization.
Of the $U$ states, a subset of $V \le U$ states are sentinel states, $1 \le i_1 < \cdots i_V \le U$,
for which we observe the first hitting time statistics for the transitions between the
sentinel states.

Generalizing the theorem in the previous section, the distribution of the first hitting time
statistic, $T_{i_v \rightarrow i_{v\pm1}}$, for the observed time to transition from either
sentinel states $i_v < i_{v+1}$, or $i_{v-1} < i_v$ is then
\begin{IEEEeqnarray*}{rCl}
	p_G\left(T_{i_v \rightarrow i_{v \pm 1}} = t\right)
		& = & \left\langle \hat{e}_{i_v}, \sum_{j=i_v}^{i_{v \pm 1} \mp 1}P_jG \exp\left(t\sum_{j=i_v}^{i_{v \pm 1} \mp 1}P_jG\right) \hat{e}_{i_{v \pm 1}} \right\rangle\\
		& = & \left\langle \hat{e}_{i_v}, \left(x_{i_v\left(i_v - 1\right)} C_{i_v\left(i_v - 1\right)} + x_{i_v\left(i_v + 1\right)} C_{i_v\left(i_v + 1\right)} \right)\exp\left(t\sum_{j=i_v}^{i_{v \pm 1} \mp 1}x_{j\left(j \pm 1\right)}C_{j\left(j \pm 1\right)}\right) \hat{e}_{i_{v \pm 1}} \right\rangle\\
		& = & \left\langle x_{i_v \left(i_v \pm 1\right)} \hat{e}_{i_v \pm 1} - \hat{e}_{i_v} \left(x_{i_v \left(i_v - 1\right)} + x_{i_v \left(i_v + 1\right)}\right),  \exp\left(t\sum_{j=i_v}^{i_{v \pm 1} \mp 1}x_{j\left(j \pm 1\right)}C_{j\left(j \pm 1\right)}\right) \hat{e}_{i_{v \pm 1}} \right\rangle 
\end{IEEEeqnarray*}
The next step is to formulate the log-likelihood. This requires establishing the observed
data. Consider $N$ observations of the durations $t_n$ between sentinel states $i_{v_n}$ and
$i_{v_n \pm 1}$, where $1 \le n \le N$. Defining the per observation generator $G_n$ as
\begin{IEEEeqnarray*}{rCl}
	G_n
		& = & \sum_{j=i_{v_n}}^{i_{v_n \pm 1} \mp 1}x_{j\left(j \pm 1\right)}C_{j\left(j \pm 1\right)}
\end{IEEEeqnarray*}
and the per observation rate difference vector $\vec{e}_n$
\begin{IEEEeqnarray*}{rCl}
	\vec{e}_n
		& = & x_{i_v \left(i_{v_n} \pm 1\right)} \hat{e}_{i_{v_n} \pm 1} - \hat{e}_{i_{v_n}} \left(x_{i_{v_n} \left(i_{v_n} - 1\right)} + x_{i_{v_n} \left(i_{v_n} + 1\right)}\right)
\end{IEEEeqnarray*}
the log-likelihood follows as
\begin{IEEEeqnarray*}{rCl}
	\Lambda
		& = & \sum_{n=1}^N \ln p_G\left(T_{i_{v_n} \rightarrow i_{v_n \pm 1}} = t_n\right)\\
		& = & \sum_{n=1}^N \ln \left(\left\langle \vec{e}_n, \exp\left(t_n G_n\right) \hat{e}_{i_{v_n \pm 1}} \right\rangle \right)
\end{IEEEeqnarray*}
Maximization requires differentiation by the $2U-3$ parameters $x_{i\left(i \pm 1\right)}$.
By the linearity of the generator this will result in algebraically replacing the $x_{i\left(i \pm 1\right)}$
with indicator functions of the form $\mathbb{I}\left[i_{v_n} = k \right]$, and $\mathbb{I}\left[k \in i_{v_n}, \dots, i_{v_n \pm 1} \mp 1 \right]$, 
where $k$ is the index of partial differentiation.
\begin{IEEEeqnarray*}{rCl}
	\frac{\partial \Lambda}{\partial x_{k\left(k \pm 1\right)}}
		& = & \sum_{n=1}^N \mathbb{I}\left[i_{v_n} = k\right] \frac{\left\langle \hat{e}_{k \pm 1} - \hat{e}_k, \exp\left(t_n G_n\right) \hat{e}_{i_{v_n \pm 1}} \right\rangle}{ \left\langle \vec{e}_n, \exp\left(t_n G_n\right) \hat{e}_{i_{v_n \pm 1}} \right\rangle }\\[2ex]
		&   & \:+ t_n \mathbb{I}\left[k \in i_{v_n}, \dots, i_{v_n \pm 1} \mp 1 \right] \frac{\left\langle \vec{e}_n, \left[ \frac{e^{\operatorname{ad}_{t_nG_n} \cdot} -1}{\operatorname{ad}_{t_nG_n} \cdot } \right]\left(C_{k\left(k \pm 1 \right)}\right) \exp\left(t_n G_n\right) \hat{e}_{i_{v_n \pm 1}} \right\rangle}{ \left\langle \vec{e}_n, \exp\left(t_n G_n\right) \hat{e}_{i_{v_n \pm 1}} \right\rangle }\\[2ex]
		& = & 0
\end{IEEEeqnarray*}
Extracting the vector differences in the previous partial differential equation to either
side of the equality yields an pseudo-linear equation that can be used to generate a 
recursive solver for the parameters $x_{i\left(i \pm 1\right)}$.
\clearpage
\section{Figures and Illustrations}
\begin{figure}[!ht]
	\centering
	\begin{tikzpicture}[->,thick,node distance=2cm]
		\node[state, very thick, font=\bf] (healthy)                                        {$1$};
		\node[state, draw=gray, text=gray] (healthy-improving) [right of=healthy]           {$2$};
		\node[state, draw=gray, text=gray] (healthy-worsening) [right of=healthy-improving] {$3$};
		\node[state, very thick, font=\bf] (treated)           [right of=healthy-worsening] {$4$};
		\node[state, draw=gray, text=gray] (treated-improving) [right of=treated]           {$5$};
		\node[state, draw=gray, text=gray] (treated-worsening) [right of=treated-improving] {$6$};
		\node[state, very thick, font=\bf] (death)             [right of=treated-worsening] {$7$};
		\path
			(healthy)           edge [loop left, very thick, font=\bf]  node         {$-x_{12}$}                     (healthy)
			(healthy)           edge [bend left, very thick, font=\bf]  node [above] {$x_{12}$}                      (healthy-improving)
			(healthy-improving) edge [bend left, very thick, font=\bf]  node [below] {$x_{21}$}                      (healthy)
			(healthy-improving) edge [loop below, draw=gray, text=gray] node         {$-\left(x_{21}+x_{23}\right)$} (healthy-improving)
			(healthy-improving) edge [bend left, draw=gray, text=gray]  node [above] {$x_{23}$}                      (healthy-worsening)
			(healthy-worsening) edge [bend left, draw=gray, text=gray]  node [below] {$x_{32}$}                      (healthy-improving)
			(healthy-worsening) edge [loop above, draw=gray, text=gray] node         {$-\left(x_{32}+x_{34}\right)$} (healthy-worsening)
			(healthy-worsening) edge [bend left, very thick, font=\bf]  node [above] {$x_{34}$}                      (treated)
			(treated)           edge [bend left, very thick, font=\bf]  node [below] {$x_{43}$}                      (healthy-worsening)
			(treated)           edge [loop below, very thick, font=\bf] node         {$-\left(x_{43}+x_{45}\right)$} (treated)
			(treated)           edge [bend left, very thick, font=\bf]  node [above] {$x_{45}$}                      (treated-improving)
			(treated-improving) edge [bend left, very thick, font=\bf]  node [below] {$x_{54}$}                      (treated)
			(treated-improving) edge [loop above, draw=gray, text=gray] node         {$-\left(x_{54}+x_{56}\right)$} (treated-improving)
			(treated-improving) edge [bend left, draw=gray, text=gray]  node [above] {$x_{56}$}                      (treated-worsening)
			(treated-worsening) edge [bend left, draw=gray, text=gray]  node [below] {$x_{65}$}                      (treated-improving)
			(treated-worsening) edge [loop below, draw=gray, text=gray] node         {$-\left(x_{65}+x_{67}\right)$} (treated-worsening)
			(treated-worsening) edge [very thick, font=\bf]             node [above] {$x_{67}$} (death);
	\end{tikzpicture}
	\caption[$7$ State Aging Process]{A representation of an aging process by a reversible $7$ state birth-death process, with $3$ sentinel states: healthy ($1$), care placement ($4$), and death ($7$). Each pair of intermediate states represents either a state of improving ($2$, $5$) or worsening ($3$, $6$) health.}
	\label{fig:agingprocess}
\end{figure}