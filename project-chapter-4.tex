\chapter{Maximum Likelihood Estimation from First Hitting Times}
\section{Distribution of First Hitting Times}
Contemporary methods for fitting time homogeneous Markov processes on a finite 
state space require directly parameterizing the transition probability matrix $\mathbb{P}\left[X_n = j \left\|X_0 = i \right.\right] = \hat{e}_i P^n \hat{e}_j$,
as they depend on realizing the process through discrete time steps $n$. While this 
formulation has many powerful applications, there are analyses where the parameterization of 
the generator of the time homogeneous Markov process, $\mathbb{P}\left[X_t = j \left\|X_0 = i \right.\right] = \hat{e}_i \exp\left({tG}\right) \hat{e}_j$, 
is of greater meaning, or importance. In particular when the observed process is, at least,
in principle continuous, or when the desired parameterization is in units of rates per time
parameterization of the generator is the more natural choice.

%With total knowledge of all first hits p_ij=N_ij/T_ij
%Prove its a stopped process where we take the product with the projection onto 
%all the other transitative states. Essentially zero out the rows of the states
%we observe first hitting on, turn them into absorbing states
% Introduce simple linear parameterization of the vector space of the algebra
\section{The Likelihood and Its Maximization}
%Long formula basically
\section{Newton-Raphson Maximization}
\subsection{Formulation}
\subsection{Algorithm}
%SIMD map-reduce version of looping through the observations